\documentclass[letterpaper]{article}
\usepackage{enumerate}
\usepackage[hmargin=3.0cm,vmargin=2.5cm]{geometry}

\begin{document}

\begin{center}
\textsc{\LARGE Juan Nicol\'as Garavito Camargo}\\
\textsc{\large Curriculum Vitae}\\
\end{center}

$    $


\textsc{\Large Contact Information:}\\
{\bf---------------------------------------------------------------------------------------------------------------------}
Calle 18A\# 1- 10 \indent \ \ \ \ \ \ \ \ \ \ \ \ \ \ \ \ \ \ \ \ \ \ \ \ \ \ \ \ \ \ \ \ \ \  \ \  \ {\bf Email:} jn.garavito57 [at] uniandes [dot] edu [dot] co       
Bloque I, Of 115 \indent \ \ \ \ \ \ \ \ \ \ \ \ \ \ \ \ \ \ \ \ \ \ \ \ \ \ \ \ \ \ \ \ \ \ \ \ \ \ \ \ \ \ \ \ \ \ \ \ \ \ \ \ \ \ \ \ \ \ \ \ \ \    http://github.com/jngaravitoc            
Universidad de los Andes\\
AA 4976, Bogot\'a, Colombia.\\
	
\indent \textsc{\Large Personal Details:}\\
{\bf---------------------------------------------------------------------------------------------------------------------}
{\bf \large Nationality:} \large Colombian-Venezuelan.\\
{\bf Date of Birth:} 02/October/1988, Age 25.\\
{\bf Marital Status:} Single.\\
{\bf Languages}: Spanish (Native), English (Fluent).\\

\textsc{\Large Education:}\\
{\bf---------------------------------------------------------------------------------------------------}
M.Sc., Physics: Universidad de los Andes (Colombia).\indent \ \ \ \ \ \ \ \ \ \ \ \ \ \ \  8/2013-present
(Advisor: Dr. Jaime E. Forero-Romero)\\
B.Sc., Physics: Universidad Nacional de Colombia. \indent \ \ \ \ \ \ \ \ \ \ \ \ \ \ \ \ \ \ \ \ \ \ \ \ \ \ \ \ \ \ \ \ \ 2/2013\\
(Advisor: Dr. Rigoberto A. Casas Miranda)\\

\textsc{\Large Research Interests:}\\
{\bf---------------------------------------------------------------------------------------------------}
Galaxy Formation, The Local Universe, High Redshift Galaxies, Reionization,  N-body/SPH Simulations, Data Mining 
and Visualization.\\

\textsc{\Large Computational Skills}:\\
{\bf---------------------------------------------------------------------------------------------------}
Programming Languages: Python, C, C++.\\
Software Packages: IRAF, ds9, Mathematica, \LaTeX.\\
Operating Systems: Unix, Linux, MacOS, OSX, Windows.\\
Visualization Software: TOPCAT, GLUE, Sphviewer.\\

\textsc{\Large Awards:}\\
{\bf---------------------------------------------------------------------------------------------------}
2013 - Best Project at GAIA summer school held at Mexico City.\\
2010 - $1^{st}$ Place Maloka (interactive Science Museum) contest to design a didactical physics experiment.\\ 

\textsc{\Large Congresses \& Meetings:}\\
{\bf---------------------------------------------------------------------------------------------------}
$\left[6\right]$ Satellite galaxies and dwarfs in the local group. Leibniz Institute for Astrophysics, Potsdam, Germany. \indent \indent \ \ \ \ \ \
\ \ \ \ \ \ \ \ \ \ \ \ \ \ \ \ \ \ \ \ \ \ \ \ \ \ \ \ \ \ \ \ \ \ \ \ \ \ \ \ \ \ \ \ \ \ \ \ \ \ \ \ \ \ \ \ \ \ \ \ \ \ \ 08/2014  \\
$\left[5\right]$ European Week of Astronomy and Space Science EWASS.  \indent \ \ \ \ \ \ \ \ \ \ \ \ \ \ \ \ \ 07/2014 \\
$\left[4\right]$ XIV Latin American Regional IAU Meeting (LARIM), Florianopolis, Brasil.  11/2013
\normalsize {(Poster Contribution: 'Effects of the gas bulk Rotation on the morphology of the Ly $\alpha$ line')}\\
$\left[3\right]$\large{ Fifth Venezuelan Astronomy Reunion (RVA), Merida, Venezuela. \indent \ \ \ \  \ \ \ \ \ \ \ \ \ \ \ \ 12/2012}
\normalsize{(Talk Contribution: 'Effect of the rotation of distant galaxies on the Lyman-alpha emission line'.)}\\
$\left[2\right]$\large{ Colombian Congress in Astronomy (COCOA), Bucaramanga, Colombia. \indent \ \ 11/2012} 
\normalsize{(Poster Contribution: Study Of The Dynamical Friction Time-Scale Using N-body Simulations.)}\\
$\left[1\right]$\large{ XVII National Physics Congress, Santa Marta, Colombia. \indent \ \ \ \ \ \ \ \ \ \ \ \ \ \ \ \ \  \ \ \ \ \ \ \ \ \ \ \ \ \ \ \ \ \ \ \ \   /2009}\\

\textsc{\Large Schools \& Workshops:}\\
{\bf---------------------------------------------------------------------------------------------------}
$\left[12\right]$ CLUES meeting, Potsdam, Germany. \indent \ \ \ \ \ \ \ \ \ \ \ \ \ \ \ \ \ \ \ \ \ \ \ \ \ \ \ \ \ \ \ \ \ \ \ \ \ \ \ \ \ \ \ \ \ \ \ 08/2014\\
$\left[11\right]$ GAIA Visualization workshop, Viena, Austria. \indent \ \ \  \ \ \ \ \ \ \ \ \ \ \ \ \ \ \ \ \ \ \ \ \ \ \ \ \ \ \ \ \ \ 07/2014\\
$\left[10\right]$ Vatican Observatory Summer School (VOSS).\\
"Galaxies Near and Far, Young and Old". \indent \ \ \ \  \ \ \  \ \ \  \ \ \  \ \  
 \ \ \ \ \ \ \ \ \ \ \ \ \ \ \ \ \ \ \ \ \ \ \ \ \ \ \ \ 06/2014 \\
$\left[9\right]$ Galactic Dynamics in the Times of GAIA and other Great Surveys. \indent \ \ \ 11/2013
Mexico City, UNAM CU.\\
$\left[8\right]$ \textsc{NEBULATOM}, "A Capacity Development Workshop for Latin\\ 
American Astronomers on Emission Line Objects in the Universe". \indent \ \ \ \ \ \ \ \ \ \ \ \ \ \ \ \ \ \ 03/2013 \\
Choron\'i, Venezuela.  \\
$\left[7\right]$ III Venezuelan Astronomy School (EVA). Merida, Venezuela. \indent \ \ \ \ \ \ \ \ \ \ \ \ \ \ 12/2012\\
$\left[6\right]$ Hippac Summer School on Astrocomputing. \indent \ \ \ \ \ \ \ \ \ \ \ \ \ \ \ \ \ \ \ \ \ \ \ \ \ \ \ \ 07/2012 
University of California San Diego. \\
$\left[5\right]$ Data reduction in Cosmology School.  \indent \ \ \ \ \ \ \ \ \ \ \ \ \ \ \ \ \ \ \ \ \ \ \ \ \ \ \ \ \ \ \ \ \ \ \ \ \ \ \ \ \ \ 12/2011 
National Astronomical Observatory of Colombia.\\
$\left[4\right]$ XVI Ciclo de cursos especiais em Astronomia.  \indent \ \ \ \ \ \ \ \ \ \ \ \ \ \ \ \ \ \ \ \ \ \ \ \ \ \ \ 10/2011
National Observatory of Rio de Janeiro, Brasil. \\
$\left[3\right]$ $1^{st}$ Workshop on Statistical Physics. \indent \ \ \ \ \ \ \ \ \ \ \ \ \ \ \ \ \ \ \ \ \ \ \ \ \ \ \ \ \ \ \ \ \ \ \ \ \ \ \ \ \ \ \ \ \ 08/2011
National University of Colombia, Andes University, Bogot\'{a}, Colombia. \\
$\left[2\right]$ School of Numerical Relativity. \indent \ \ \ \ \ \ \ \ \ \ \ \ \ \ \ \ \ \ \ \ \ \ \ \ \ \ \ \ \ \ \ \ \ \ \ \ \ \ \ \ \ \ \ \ \ \ \ \ \ \ \ \ \ \ \ \ \ \ /2011
National Astronomical Observatory of Colombia.\\
$\left[1\right]$ Extragalactic Astrophysics School. \indent \ \ \ \ \ \ \ \ \ \ \ \ \ \ \ \ \ \ \ \ \ \ \ \ \ \ \ \ \ \ \ \ \ \ \ \ \ \ \ \ \ \ \ \ \ \ \ \ \ \ \ \ \ \ \ \ \ \ \ /2010
National Astronomical Observatory of Colombia.\\

\textsc{\Large Talks:}\\
{\bf---------------------------------------------------------------------------------------------------}
$\left[4\right]$ "The effect of gas bulk rotation on the Lyman Alpha line profile", EWASS, Geneve, Switzerland, July-2014.\\
$\left[3\right]$  "The most distant galaxies in the Universe", Universidad Nacional de Colombia, 2014.\\
$\left[2\right]$ "GAIAs View on the metallicities of the Milky Way", UNAM, Mexico City, Mexico, Nov-2013. \\
$\left[1\right]$  A Bayesian Method to study galaxy streams in the Milky Way Disk", CIDA, Merida, Venezuela , July-2013.\\

\textsc{\Large Publications:}\\
{\bf---------------------------------------------------------------------------------------------------}
[1] {\bf Garavito-Camargo J.N} ,Forero-Romero J.E, Dijkstra M, 2014. "Effects of gas bulk rotation on the morphology of the Lyman-alpha line", Acepted for publication to ApJ.\\

\textsc{\Large Teaching Experience:}\\
{\bf---------------------------------------------------------------------------------------------------}
Physics I Laboratory: Universidad de los Andes. \indent \ \ \ \ \ \ \ \ \ \ \ \ \ \ \ \ \ \ \ \ \ \ \ \ \ \ \ $2^{nd}$ Semester-2013\\
Computational Tools: Universidad de los Andes. \indent \ \ \ \ \ \ \ \ \ \ \ \ \ \ \ \ \ \ \ \ \ \ \ \ \ \ \ \ \ \ $1^{st}$ Semester-2014\\
Computational Methods Laboratory: Universidad de los Andes. \indent  \ \ $1^{st}$ Semester-2014\\
TA, Computational Methods: Universidad de los Andes. \indent \ \ \ \ \ \ \ \ \ \ \ \ \ \   $1^{st}$ Semester-2014\\
Computational Tools: Universidad de los Andes. \indent \ \ \ \ \ \ \ \ \ \ \ \ \ \ \ \ \ \ \ \ \ \ \ $2^{nd}$ Semester-2014\\
Computational Methods Laboratory: Universidad de los Andes. \indent  \ \ $2^{nd}$ Semester-2014\\
TA, Computational Methods: Universidad de los Andes. \indent \ \ \ \ \ \ \ \ \ \ \ \ \ \   $2^{nd}$ Semester-2014\\

\textsc{\Large Public Service:}\\
{\bf---------------------------------------------------------------------------------------------------}
Organizer of the student astronomy seminar at the Planetarium of Bogot\'a (Spacio).
\end{document}
