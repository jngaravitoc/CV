\documentclass[UTF8]{article}
\usepackage{enumerate}
\usepackage[hmargin=2cm,vmargin=2.5cm]{geometry}
\usepackage[colorlinks=true]{hyperref}
\usepackage{url}
\usepackage{etaremune}
\usepackage{fancyhdr}
\usepackage{lastpage}
\usepackage[dvipsnames]{xcolor}
\usepackage{sectsty}



\hypersetup{
  colorlinks,
  linkcolor={red!50!black},
  citecolor={blue!50!black},
  urlcolor={cyan!90!black}}


\pagestyle{fancy}
\fancyhf{}
\lhead{Nicol\'as Garavito-Camargo}
\chead{Curriculum Vitae}
\rhead{\thepage}
\thispagestyle{empty}
%sectionfont{\color{}}

\fancyfoot[C]{\textit{Last updated: \today}}

\begin{document}

\indent \textbf{\LARGE Nicol\'as Garavito-Camargo -- Curriculum Vitae} \\
\indent \rule{17cm}{0.4pt}\\

\noindent Steward Observatory, University of Arizona, 933 North Cherry Av. Tucson, AZ.\\
E-mail: jngaravitoc@email.arizona.edu\\ 
Website: \href{http://jngaravitoc.github.io/Garavito-Camargo}{jngaravitoc.github.io}\\
Github: \href{http://www.github.com/jngaravitoc}{github.com/jngaravitoc}\\
%Phone: +1(520)265-6092
\section*{Research Interests}
Galaxy Dynamics, Local Group Dynamics, Dwarf Galaxies, Galactic
Archaeology, Astrophysical probes of Dark Matter, Galaxy Formation and Evolution.

\section*{Education}
\textbf{Ph.D.,} Astronomy and Astrophysics, University of Arizona, \textit{Expected May 2021}.\\
\indent \textit{Advisor: Dr. Gurtina Besla}\\
\textbf{M.Sc.,}  Physics, Universidad de Los Andes, Bogot\'a, Colombia, 2015.\\
\indent \textit{Advisor: Dr. Jaime E. Forero-Romero}\\
\textbf{B.Sc.,} Physics, Universidad Nacional de Colombia, Bogot\'a, Colombia, 2013.\\
\indent \textit{Advisor: Dr. Rigoberto A. Casas Miranda}

\section*{Fellowships and Awards}

University of Arizona, Theoretical Astrophysics Program, Graduate Student
Research Prize 2021.\\
University of Arizona College of Science award for Excellence in
Service for graduate students, 2020. \\
\indent \textit{(Awarded to one graduate student across the college of
science per year.)}\\
University of Arizona theory travel grant, 2016, 2018. \\
%SpaceArt scholarship, Bogot\'a Planetarium: For developing art and science material for children, 2014. \\
%Best research project at the Gaia Summer school, UNAM, Mexico City, 2013.
%2010 - $1^{st}$ Place Maloka (interactive Science Museum) contest to design a didactical physics experiment.\\
\section*{Talks}

26 Total: 17 in North America, 5 in Europe, 4 in Latin America.

\subsection*{Conferences}

\begin{itemize}
  \setlength\itemsep{0.0em}
  \renewcommand\labelitemi{$\cdot$}


\item \href{https://aas.org/meetings/dda52}{Division on Dynamics Astronomy}, Virtual meeting, May 2021.
\item \href{https://stellarstreams.org/streams21/}{Streams 21}, Virtual
  Conference, February 2021. 
\item  \href{https://www.stsci.edu/contents/events/stsci/2020/april/the-local-group-assembly-and-evolution?page=2&filterUUID=6fedb8a7-}{The Local Group: Assembly and Evolution}, virtual conference, August 2020.
\item \href{https://eas.unige.ch/EAS2020/}{European Astronomical Society meeting}, virtual meeting, June 2020.
\item Durham University, UK, \href{http://astro.dur.ac.uk/cosmodwarfs/}{Small Galaxies Cosmic Questions}, August, 2019.
\item MPIA, Heidelberg, \href{http://www.mpia.de/homes/stellarhalos2018-loc/sh2018/index.html}{Stellar halos across the cosmos}, July 2018.
\item LARIM, Cartagena, Colombia, October 2016.
\item EWASS, Geneve, Switzerland, July 2014.
\end{itemize}
  
\subsection*{Seminars}
\begin{itemize}
  \setlength\itemsep{0.0em}
  \renewcommand\labelitemi{$\cdot$}

\item \href{https://comscicon.com/comscicon-en-espa%C3%B1ol-2021}{ComSciCon}, June 2021, \dag. 
\item Steward Observatory, Galaxy lunch talk, March, 2021.
\item Univesidad de los Andes, Astronomy Seminar, February, 2021. 
\item UC Irvine, Astronomy Seminar, January, 2021. \dag
\item Steward Observatory, Early Career Scientist talk, December, 2020. \dag
\item CCAPP, Seminar, December, 2020. \dag
\item Princeton, Journal Club, November 2020. \dag
\item KIPAC, Stanford, Tea-Talk, October 2020.
\item Harvard Center for Astrophysics, GCSP seminar, October 2020.
\item Carnegie Observatories, seminar, October 2020 \dag.
\item University of California Berkeley, lunch talk, September 2020. 
\item The Royal Observatory of Edinburgh, UK, Coffee Talk, August 2019 \dag.
\item W.M. Keck Observatory, Journal Club, June 2019.
\item Magellanic Cloud Fest III, University of Arizona, May, 2019.
\item JILA Seminar, JILA Institute, University of Colorado, December 2017.
\item STScI Galaxies Journal Club. December 2016.
\item Magellanic Cloud Fest II, University of Arizona, March 2016.
\item UNAM, Mexico City, Mexico, Nov 2013. 
\item Centro de Investiaciones De Astronom\'ia CIDA, Merida, Venezuela, July 2013.
\end{itemize}
\indent \dag \textit{Invited}

\section*{Posters}
\begin{itemize}
\setlength\itemsep{0.0em}
\renewcommand\labelitemi{$\cdot$}
  \item European Astronomical Society meeting, virtual meeting, June 2020.
  \item Rediscovering our Galaxy, IAU symposium 334, Potsdam, Germany. July 2017.
\end{itemize}

\section*{Grants}
\begin{itemize}
\setlength\itemsep{0.0em}
\renewcommand\labelitemi{$\cdot$}
\item LSSTC Grant Award \#2021-51, 2001. PI: Andres A. Plazas Malag\'on
\item University of Arizona theory travel grant, 2016, 2018. 
\item IAU travel grant.2016.
\end{itemize}

\section*{Telescope Time Awarded}
\begin{itemize}
  \setlength\itemsep{0.0em}
  \renewcommand\labelitemi{$\cdot$}
\item Blanco Telescope, ``\textit{ A VISTA-DECam Experiment in Near-Field
  Cosmology: Search for the Magellanic Dark Matter Wake}" cycle 2020B. PI: Julio Chaname, 3 nights 
\end{itemize}

\section*{Professional Experience}
\textit{Computational:}\\
Esperience with C, C++, Python, OpenMP, HPC systems.\\

\textit{Service:} 
\begin{itemize}
  \setlength\itemsep{0.0em}
  \renewcommand\labelitemi{$\cdot$}
\item Co-creator of the 10-week  \href{https://recastronomia.github.io/internship/}{internship program} for astronomy students in Colombia. 2021.

\item Co-creator of the \href{https://recastronomia.github.io/mentores/}{mentorship program} for astronomy students in Colombia. 2020-present.
  (The mentorship program pairs up students with professional astronomers to
  provide guidance thorugth the application process to gradute programs)

\item Referee for the \textit{Monthly Notices of the Royal Astronomical Society} 2019-present.
\item Co-organizer,  \href{https://www.as.arizona.edu/diversity_coffee/}{Diversity Journal Club},
  Steward Observatory, 2018-present.
\item Proposal reviewer for \href{https://clubesdeciencia.co/}{Colombian Science Clubs}, 2018.
\item Local Organizing Committee of the \href{http://forero.github.io/AndeanCosmologySchool/}{Andean Cosmology School}, Universidad de Los Andes, 2015.
\item Organizer of the student astronomy seminar at the Planetarium of Bogot\'a, 2014-2015.
\end{itemize}

\section*{Outreach}

\begin{itemize}
  \setlength\itemsep{0.0em}
  \renewcommand\labelitemi{$\cdot$}
\item Co-creator of the \href{https://recastronomia.github.io/mentores/}{mentorship program} for astronomy students in Colombia. 2020-present.
  (The mentorship program pairs up students with professional astronomers to
  provide guidance thorugth the application process to gradute programs)
\item Creator, \href{https://astrocharlas.github.io/}{Astrocharlas},
Steward Observatory, 2018-present.
 (Spanish outreach series talks in astronomy)
\item Writer for \href{https://astrobitos.org/}{Astrobitos}, 2018-present.
\item Classroom astronomer, NOAO Project \href{https://www.noao.edu/education/astro/}{ASTRO}, 2018-2019.
\item Discussion leader, NOAO \href{http://www.teenastronomycafe.org/}{Teen Astronomy Cafe}, 2018.
\item Mentor, Tucson Initiative for Minority Engagement in Science and Technology Program \href{https://lavinia.as.arizona.edu/~timestep/}{TIMESTEP}, 2016-2018.
\item Planetarium SpaceArt mentor for Children, Bogot\'a, Colombia, 2013-2014.
%\indent\indent Unconscious Bias Workshop, 2017. \\
\end{itemize}



\section*{Teaching Experience}
\begin{itemize}
  \setlength\itemsep{0.0em}
  \renewcommand\labelitemi{$\cdot$}
\item TA, Astronomy Tutoring for
Majors \& Minors Program. University of Arizona, Spring 2019. 
\item Computational Physics, Teaching assistant, PHYS305. University of Arizona. Spring 2018.
\item Computational Tools: Universidad de los Andes, $1^{st}$ 2014-2015.
\item Computational Methods Laboratory: Universidad de los Andes, $1^{st}$ 2014-2015.
\item TA, Computational Methods: Universidad de los Andes, $2^{nd}$ 2014.
\item Physics I Laboratory: Universidad de los Andes, $2^{nd}$ Semester-2013.
\end{itemize}

\section*{Observing Experience}
\begin{itemize}
  \setlength\itemsep{0.0em}
  \renewcommand\labelitemi{$\cdot$}
\item DECam, Blanco-4m telescope, CTIO, Chile. 3 nights, 2020.
\item VATT telescope. Mt Graham, Arizona - 4 nights, 2016.
\item CIDA, Merida, Venezuela - 2 nights, 2013.
\end{itemize}

\section*{Publications list}

\noindent \href{https://orcid.org/0000-0001-7107-1744}{ORCID},
\href{https://ui.adsabs.harvard.edu/search/q=docs(library%2F0X5_bcuLT4iE-6-Nko0kmg)&sort=date%20desc%2C%20bibcode%20desc&p_=0}{ADS},
\href{https://arxiv.org/search/?query=garavito-camargo&searchtype=all}{arXiv}\\
refereed: 11 -- submitted: 1 -- first author: 4 -- $h-$index: 8 -- citations :
292 (as of June 18th/2021) 
\begin{etaremune}

\item \textit{The Clustering of Orbital Poles Induced by the LMC: Hints for
      the Origin of Planes of Satellites}\\ 
      \textbf{Garavito-Camargo, Nicol\'as}; Patel, Ekta; Besla, Gurtina; Price-Whelan,
      Adrian; Laporte, Chervin; G\'omez, Facundo A; Kathryn V. Johnston; ApJ
      submitted (2021). 

\item \textit{Detection of the All-Sky Response of the Galactic
  Halo to the Magellanic Clouds}\\ 
  Conroy, Charlie; Naidu, Rohan P; \textbf{Garavito-Camargo; Nicol\'as}; Besla,
  Gurtina; et al. Nature (2021). 

\item \textit{Quantifying the impact of the Large Magellanic Cloud on the
  structure of the Milky Way’s dark matter halo using Basis Function Expansions}\\ 
  \textbf{Garavito-Camargo, Nicol\'as}; Besla, Gurtina; Laporte,
  Chervin; Price-Whelan, Adrian M.; et al. ApJ accepted (2021). 

\item \textit{Quantifying the Stellar Halo's Response to the LMC's Infall with
  Spherical Harmonics}.\\
  Cunningham, Emily C; \textbf{Garavito-Camargo, Nicolas}, Deason, Alis J;
  Johnston, Kathryn V. et al. ApJ, 898, 1,(2020).

\item \textit{The orbital histories of Magellanic Satellites Using Gaia DR2
  proper motions}. \\
  Patel, Ekta; Kallivayalil, Nitya; \textbf{Garavito-Camargo, Nicolas} et al.
  ApJ 893, 121, (2020).
\item \textit{The highest-speed local dark matter particles come from the Large
  Magellanic Cloud}. \\
  Besla, Gurtina; Peter, Annika; \textbf{Garavito-Camargo, Nicolas}. JCAP, 11,
  13, (2019).
\item \textit{Hunting for the DM Wake induced by the LMC}.\\
  \textbf{Garavito-Camargo, Nicolas}; Besla, Gurtina; Laporte, Chervin F.P;
  Johnston, Kathryn V; G\'omez, Facundo A; Watkins, Laura. ApJ Accepted, (2019).
\item \textit{The influence of Sagittarius and the Large Magellanic Cloud on the
  stellar disc of the Milky Way Galaxy.}\\
  Laporte, Chervin F. P; Johnston, Kathryn V; G\'omez, Facundo A; \textbf{Garavito-Camargo, Nicolas}; Besla,
  Gurtina. MNRAS, 481, 286L, (2018).
\item \textit{The Extremely Luminous Quasar Survey in the Sloan Digital Sky
  Survey Footprint. II. The North Galactic Cap Sample.}\\ Schindler, Jan-Torge;
  Fan, Xiaohui; McGreer, Ian D; Yang, Jinyi; Wang, Feige; Green, Richard;
  \textbf{Garavito-Camargo, Nicolas} et al., ApJ, 863, 144S, (2018).
\item \textit{Modelling the gas kinematics of an atypical Ly$\alpha$
emitting compact dwarf galaxy.}\\  Forero-Romero, Jaime E., Gronke, Max.,
Remolina-Gutiérrez, Maria Camila, \textbf{Garavito-Camargo, Nicolas}, Dijkstra
M. MNRAS, 474, 12F, (2018).
\item \textit{Response of the Milky Way's disc to the Large Magellanic Cloud in
  a first infall scenario.}\\ Laporte, C.; Gomez, F; Besla, Gurtina; Johnston,
  Kathryn V; \& \textbf{Garavito-Camargo, Nicolas}. MNRAS, 473, 1218L, (2018).
 \item \textit{The impact of gas bulk rotation on the morphology of the
   Lyman-alpha line}.\\ \textbf{Garavito-Camargo, J.N}; Forero-Romero J.E;
   Dijkstra M. ApJ, 795, 120, (2014).
\end{etaremune}




\section*{White papers}

\begin{etaremune}
\item \textit{Mass Spectroscopy of the Milky Way} \\
  Dey, Arjun. et al. (incl. \textbf{Garavito-Camargo}).
  \href{https://113qx216in8z1kdeyi404hgf-wpengine.netdna-ssl.com/wp-content/uploads/2019/05/489_dey.pdf}{Astro2020: Decadal
  Survey on Astronomy and Astrophysics, Vol. 51, Issue 3, id. 489 (2019).}
\item \textit{The Multidimensional Milky Way.}\\
 Sanderson, Robyn E .et al. (incl. \textbf{Garavito-Camargo}) 2019.
  \href{https://arxiv.org/abs/1909.07641}{Astro2020: Decadal Survey on
Astronomy and Astrophysics, 2019, Vol. 51, Issue 3, id. 347 (2019).} 
\end{etaremune}


\end{document}


