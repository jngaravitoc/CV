\documentclass[14pt]{article}
\usepackage{enumerate}
\usepackage[hmargin=2cm,vmargin=2cm]{geometry}
\usepackage[colorlinks=true]{hyperref}
\usepackage{url}
\usepackage{etaremune}
\usepackage{fancyhdr}
\usepackage{lastpage}
\usepackage[dvipsnames]{xcolor}
\usepackage{sectsty}
\usepackage{cmtt}

\usepackage{xcolor}
\usepackage{hyperref}

\usepackage[T1]{fontenc}


\hypersetup{
  colorlinks=True,
  linkcolor={red!50!black},
  citecolor={blue!50!black},
  urlcolor={Purple}}


\pagestyle{fancy}
\fancyhf{}
\lhead{Nicol\'as Garavito-Camargo}
\chead{Curriculum Vitae}
\rhead{\thepage}
\thispagestyle{empty}
%sectionfont{\color{}}

\fancyfoot[C]{\textit{Last updated: \today}}

\begin{document}
\begin{center}
\indent \textbf{\LARGE Nicol\'as Garavito-Camargo -- Curriculum Vitae} \\
\indent \rule{17cm}{0.4pt}\\
\end{center}

\fontsize{11}{11}\selectfont

\begin{center}
  Barbara Pichardo Future Faculty Fellow $|$
  \href{https://astro.arizona.edu/}
  {University of Arizona}, Tucson, AZ\\
E-mail: nico.garavito@gmail.com $|$ Website: \href{http://jngaravitoc.github.io/Garavito-Camargo}{jngaravitoc.github.io} $|$ Github: \href{http://www.github.com/jngaravitoc}{github.com/jngaravitoc}\\
\end{center}

\begin{center}
\textit{\textbf{Research Interests:} Galactic Dynamics -- Astrophysical probes
  of Dark Matter -- Computational Methods -- High-Performance Computing --
N-body Simulations -- Software development}\\
\end{center}

%Phone: +1(520)265-6092
%\section*{Research Interests}
%Galaxy Dynamics, Local Group Dynamics, Dwarf Galaxies, Galactic
%Archaeology, Astrophysical probes of Dark Matter, Galaxy Formation and Evolution.



\section*{Appointments}

\indent Barbara Pichardo Future Faculty Fellow, University of Arizona, February
2025-August 2025. \\

\noindent NFHP Einstein Fellow, University of Maryland, September 2026 - August 2029. 


\section*{Education and past positions}


\indent Flatiron Research Fellow, Flatiron Institute, October 2021-December 2024.\\

\textbf{Ph.D.,} Astronomy and Astrophysics, \href{https://www.as.arizona.edu/}{University of Arizona}, 2021.\\
\indent \textit{Advisor: \href{https://sites.google.com/view/thebeslagroup/home}{Dr. Gurtina Besla}}\\

\textbf{M.Sc.,}  Physics, \href{https://fisica.uniandes.edu.co/en}{Universidad de Los Andes}, Bogot\'a, Colombia, 2015.\\
\indent \textit{Advisor: \href{http://wwwprof.uniandes.edu.co/~je.forero/}{Dr. Jaime E. Forero-Romero}}\\

\textbf{B.Sc.,} Physics,
\href{https://unal.edu.co/en.html}{Universidad Nacional de Colombia}, Bogot\'a, Colombia, 2013.\\
\indent \textit{Advisor:
\href{https://www.iau.org/administration/membership/individual/16146/}{Dr.
Rigoberto A. Casas Miranda}}



\section*{Scholarships and Awards}


\begin{itemize}
  \setlength\itemsep{0.0em}
  \renewcommand\labelitemi{$\cdot$}

\item Barbara Pichardo Future Faculty Fellowship University of Arizona 2025-2027. 
\item University of Arizona, Theoretical Astrophysics Program, Graduate Student
Research Prize 2021.
\item University of Arizona College of Science award for Excellence in
Service for graduate students, 2020.\\ 
\indent \textit{(Awarded to one graduate student across the college of
science per year.)}
\item University of Arizona theory travel grant, 2016, 2018. 
\item McCarthy-Stoeger Scholarship 2015-2017, Vatican Observatory.
\end{itemize}
%SpaceArt scholarship, Bogot\'a Planetarium: For developing art and science material for children, 2014. \\
%Best research project at the Gaia Summer school, UNAM, Mexico City, 2013.
%2010 - $1^{st}$ Place Maloka (interactive Science Museum) contest to design a 
%didactic physics experiments.\\


\section*{Students supervision}
I have had the privilege of advising or co-advising a total of eight students
(five Ph.D., three undergraduates) from diverse backgrounds. 


\begin{itemize}
  \setlength\itemsep{0.0em}
  \renewcommand\labelitemi{$\cdot$}
  \item \textbf{Richard Brooks} (Graduate Student at University of College
    London),Spring 2024 - present. I am advising Richard on one research project that
    was submitted for publication to ApJ (see article 29 in the publications
    section). We are currently working on a second paper, expected for
    submission in Spring 2025.
  \item \textbf{Elise Darragh-Ford} (Graduate student at Stanford), Fall 2023-Summer
    2024. I advised Elise on a research project that was the  final chapter of her Ph.D. thesis, 
    resulting in one paper currently under preparation.  
  \item \textbf{Silvio Varela} (Graduate student at Universdad de la Serena), Fall 2022-present. I am currently 
    co-advising Silvio on a research project that will lead to a paper now in preparation
  \item \textbf{Arpit Arora} (Graduate student at University of Pennsylvania);
    Fall 2021-present. I am currently co-advising (now Dr.) Arpit on two
    research projects: one paper has been accepted to ApJ (see article 27th in the publications section), and another is in preparation.
  \item \textbf{Hayden Foote} (Graduate student at University of Arizona), Fall
    2021-present I co-advised Hayden on a research project that resulted in a
    submitted publication (see article 28th in the publications section), and a second paper 
    is currently in preparation.
  \item \textbf{Andrew Eden} (Undergraduate at Florida Institute of Technology),
    Fall 2022-Fall 2023. I advise Andrew on his undergraduate thesis project which
    will result in a paper that is currently in preparation. All the code that
    Andrew develop for his project is publicly available on \href{https://github.com/aeden2019/rubin_mw_mocks}{Github}. 
  \item \textbf{Ludia Adhikary} (Undergraduate CUNY), supervised through the AstroCOM CUNY/CCA program; NYC Summer 2022-Summer 2023. Ludia
    presented her work at the CCA symposium and she presented two posters. One
    at the SACNAS conference in 2022 and one at the winter AAS of 2023. 
  \item \textbf{Stephanie Carolina Cely Rodriguez} (Undergraduate at Universidad
    Nacional de Colombia), I advise undergraduate Stephanie on her undergraduate thesis as part of the
   RECA summer program in  Summer 2022. Stephanie successfully graduated in
   Summer 2023. A talk contribution from Stephanie was given at the RECA
   symposium and can be watch \href{https://www.youtube.com/watch?v=5SeOW060m24&t=1774s}{here}.
\end{itemize}

\section*{Teaching Experience}

Principal lecturer of 3 courses for undergraduate level in physics for a total
of 180 hours of teaching time. Teaching assistant for 3 courses.

\begin{itemize}
  \setlength\itemsep{0.0em}
  \renewcommand\labelitemi{$\cdot$}
\item Guest Lecturer for the graduate Galaxies class, Columbia University, Fall 2022.
\item Teaching assistant for the Astronomy Tutoring for Majors \& Minors Program. University of Arizona, Spring 2019. 
\item Teaching assistant for the Computational Physics class  PHYS305. University of Arizona. Spring 2018.
\item 3 times Lecturer of the \href{https://github.com/ComputoCienciasUniandes/HerramientasComputacionales}{Computational
  Tools} at Universidad de los Andes, spring semester 2014 - spring 2015.
\item 3 times Lecturer of the computational Methods Laboratory at Universidad de los Andes, semester spring 2014 -  spring 2015.
\item Teaching assistant of the
  \href{https://github.com/ComputoCienciasUniandes/MetodosComputacionales}{Computational
  Methods} Universidad de los Andes, fall  semester 2014.
\item Lecturer (for 3 different sessions) of the class Physics I Laboratory at Universidad de los Andes, fall semester 2013.
\end{itemize}


\subsection*{Academic Service} 

\begin{itemize}
  \setlength\itemsep{0.0em}
  \renewcommand\labelitemi{$\cdot$}

\item Lead coordinator of the EXP collaboration. I coordinate the cosmological
  simulations working group. I am in charge of 3 graduate students. Other responsabilities include: organizing in person meetings twice a year.
\item Science organizing committee of the  \href{https://benasque.org/new_general/cgi-bin/years.pl?ano=2026}{Clouds over
  the Pyrenees} conference to be held in Benasque, Spain from September 6-9, 2026. 
  \item Science organizing committee of the XMC workshop in Yellowstone to be held
  in May 26th-30th 2025, at the University of Montana. 
\item Local and Science organizing committee of the \href{https://events.simonsfoundation.org/event/1f18ba51-d6da-414f-b492-9590f29ad048/summary}{Milky Clouds over Manhattan conference}. 
\item PhD. thesis committee of Dr.Silvio Varela (University of La Serena, Chile), Nov-2023. 
\item Referee for: Astrophysical Journal (ApJ), Monthly Notices of the Royal  Astronomical Society (MNRAS), Nature, Nature Astronomy, Astronomy and
  Astrophysics (A\&A), and the Galaxies Journal.
\item Beyond-BFE collaboration coordinator: I lead the cosmological simulation group, organize in person and online meetings.
\item Mentor at the City University of New York CUNY-CCA program for undergraduate students at CUNY working with mentors at the CCA, summer 2022. 
\item Proposal reviewer for \href{https://clubesdeciencia.co/}{Colombian Science Clubs}, 2018.
\item Local Organizing Committee of the \href{http://forero.github.io/AndeanCosmologySchool/}{Andean Cosmology School}, Universidad de Los Andes, 2015.
\item Organizer of the student astronomy seminar at the Planetarium of Bogot\'a, 2014-2015.
\end{itemize}


\section*{Diversity, Equity, Inclusion (DEI) and outreach leadership:}


\subsection*{DEI Leadership:}

I have worked extensively to build supportive communities. I co-created a support network for astronomy students in Colombia, \href{https://www.astroreca.org/}{RECA}, which has grown to 478 members, making it the largest student association in astronomy in Latin America. In 2020, I established a yearly mentorship program designed to support students in their careers. To date, we have worked with 141 students, each paired with a senior mentor for guidance. Additionally, we have compiled valuable resources, including career panels with guest speakers and online guides for applying to PhD programs. I have also secured funding to run this program through the IAU.


\begin{itemize}
  \setlength\itemsep{0.0em}
  \renewcommand\labelitemi{$\cdot$}
\item Co-creator of the 10-week
  \href{https://recastronomia.github.io/internship/}{RECA internship program}
  for astronomy students in Colombia. May-August 2021.
\item Co-creator of the \href{https://recastronomia.github.io/mentores/}{RECA mentorship program} for astronomy students in Colombia. 2020-2024.
  (The mentorship program pairs up students with professional astronomers to
  provide guidance through the application process to graduate programs)
\item Career panelist for CUNY undergraduates in stem ``an initiative to enable
  low-income, talented domestic students to pursue successful careers in
  promising STEM fields" organized by Professor Viviana Acquaviva. December 3rd 2021.  
\item Co-organizer,
  \href{https://www.as.arizona.edu/diversity_coffee/}{Diversity Journal Club} Steward Observatory, 2018-2021.
\item Creator, \href{https://astrocharlas.github.io/}{Astrocharlas},
Steward Observatory, 2018-2021.
 (Spanish outreach series talks in astronomy)
\item Writer for \href{https://astrobitos.org/}{Astrobitos}, 2018-2021.
\item Mentor, Tucson Initiative for Minority Engagement in Science and Technology Program \href{https://lavinia.as.arizona.edu/~timestep/}{TIMESTEP}, 2016-2018.
\end{itemize}

\subsection*{Outreach}
\begin{itemize}
  \setlength\itemsep{0.0em}
  \renewcommand\labelitemi{$\cdot$}
\item NYC Prison outreach program at MDC Brooklyn. Lead a 2 sessions in 2024 of 3-hour outreach
  session with Spanish speaking women and men. Program organizer: Kiyan Tavangar (U. Columbia) 
\item Astronomy podcast in Spanish \href{https://cosmoquest.org/x/visioncosmica/nuestro-equipo/}{vision
  cosmica}. I participated in 20 episodes talking about various topics of
  astronomy for non-specialist audiences. 
\item Classroom astronomer, NOAO Project \href{https://www.noao.edu/education/astro/}{ASTRO}, 2018-2019.
\item Discussion leader, NOAO \href{http://www.teenastronomycafe.org/}{Teen Astronomy Cafe}, 2018.
\item Planetarium SpaceArt mentor for Children, Bogot\'a, Colombia, 2013-2014.
%\indent\indent Unconscious Bias Workshop, 2017. \\
\end{itemize}



\subsection*{Open source and HPC experience}

\begin{itemize}
  \setlength\itemsep{0.0em}
  \renewcommand\labelitemi{$\cdot$}
  \item Core developer and active contributor of the python packages:
    (Core developer) \href{https://github.com/EXP-code/EXP_tools}{EXPtools}, 
    (Core developer) \href{https://github.com/jngaravitoc/nba}{NBA}, Cranes. 
    (Contributor) \href{https://github.com/athob/py-ananke}{py-Ananke},
    (Contributor) \href{https://gala.adrian.pw/en/latest/}{gala},
    (Contributor) \href{https://bitbucket.org/awetzel/halo_analysis/src/master/}{halo analysis},
    (Contributor) \href{https://bitbucket.org/awetzel/gizmo_analysis/src/master/}{gizmo analysis}.
  \item Expertise with HPC N-body codes: Gadget-3, 4, EXP (CPU \& GPU), AREPO.  

\end{itemize}
\section*{Scientific Talks}

47 Total: 18 Invited (denoted by \dag), 26 in North America, 6 in Europe, 9 in Latin America, 1 Asia.\\


\subsection*{Conferences (14)}

\begin{itemize}
  \setlength\itemsep{0.0em}
  \renewcommand\labelitemi{$\cdot$}
  \item KITP workshop ``Dark Matter Theory, Simulation, and Analysis in the Era of Large Surveys'', UC Santa Barbara, June 2024. \dag
  \item The Milky Way is not an island, Sexten, February, 2024. 
\item Surveying the Milky Way: The Universe at our backyard. Caltech,
  Pasadena, October, 2023.
\item Friends of Friends meeting, Cordoba, Argentina, April, 2023. \dag
\item IAU 379: Dynamical masses of local group galaxies, contributed talk, March, 2023.
\item \href{https://www.mso.anu.edu.au/~yting/Malaysia_IAU/}{IAU 377: Early Disk-Galaxy Formation
from JWST to the Milky Way}, contributed talk, February, 2023.
\item \href{https://accefyn.com/microsites/nodos/astroco/congreso-colombiano-de-astronomia-cocoa-2022%E2%80%8B/}{Colombian Congress of Astronomy}, Plenary talk, August, 2022. \dag
\item \href{http://fof.oac.uncor.edu/2022/}{Friends of friends meeting}, Cordoba, Argentina, April, 2022.
\item \href{https://aas.org/meetings/dda52}{Division on Dynamics Astronomy}, Virtual meeting, May 2021.
\item \href{https://stellarstreams.org/streams21/}{Streams 21}, Virtual
  Conference, February 2021. 
\item  \href{https://www.stsci.edu/contents/events/stsci/2020/april/the-local-group-assembly-and-evolution?page=2&filterUUID=6fedb8a7-}{The Local Group: Assembly and Evolution}, virtual conference, August 2020.
\item \href{https://eas.unige.ch/EAS2020/}{European Astronomical Society meeting}, virtual meeting, June 2020.
\item Durham University, UK, \href{http://astro.dur.ac.uk/cosmodwarfs/}{Small Galaxies Cosmic Questions}, August, 2019.
\item MPIA, Heidelberg, \href{http://www.mpia.de/homes/stellarhalos2018-loc/sh2018/index.html}{Stellar halos across the cosmos}, July 2018.
\item LARIM, Cartagena, Colombia, October 2016.
\item EWASS, Geneve, Switzerland, July 2014.
\end{itemize}
  
\subsection*{Seminars and Colloquia (33)}
\begin{itemize}
  \setlength\itemsep{0.0em}
  \renewcommand\labelitemi{$\cdot$}


\item UC Santa Barbara, seminar, June 2024. \dag
\item KITP UC Santa Barbara, seminar, June 2024. \dag
\item UC Riverside, seminar, October 2023. \dag
\item Universidad de La Serena, La Serena, Chile, May 2023. \dag 
\item Instituto de Astronom\'ia y F\'isica del Espacio (IAFE), Buenos Aires, April, 2023.
\item Max Planck Institute for Astrophysics, Cosmology Seminar, March, 2023.
\item U. Rutgers, Astronomy Seminar, Nov, 2022. \dag
\item U. Columbia, Lunch talk, Sept, 2022. \dag
\item STScI, galaxies lunch talk, May, 2022. \dag
\item Universidad de Ant\'ioquia, seminar, February, 2022. \dag 
\item University of Massachusetts, Amherst, Colloquium, Jan, 2022. \dag
\item University of Madison-Wisconsin, Science seminar, November 2021.\dag
\item University of Michigan, Galaxies group seminar, November 2021.\dag
\item NYU, CCPP, seminar, November 2021.\dag 
\item CCA, Flatiron Institute, Lunch Talk, October, 2021.
\item Steward Observatory, Theoretical Astrophysics Program (TAP) colloquium, September, 2021.\dag
\item \href{https://comscicon.com/comscicon-en-espa%C3%B1ol-2021}{ComSciCon},
    June 2021. \dag. 
\item Steward Observatory, Galaxy lunch talk, March, 2021.
\item Univesidad de los Andes, Astronomy Seminar, February, 2021. 
\item UC Irvine, Astronomy Seminar, January, 2021. \dag
\item Steward Observatory, Early Career Scientist talk, December, 2020. \dag
\item CCAPP, Seminar, December, 2020. \dag
\item Princeton, Journal Club, November 2020. \dag
\item KIPAC, Stanford, Tea-Talk, October 2020.
\item Harvard Center for Astrophysics, GCSP seminar, October 2020.
\item Carnegie Observatories, seminar, October 2020. \dag
\item University of California Berkeley, lunch talk, September 2020. 
\item The Royal Observatory of Edinburgh, UK, Coffee Talk, August 2019. \dag
\item W.M. Keck Observatory, Journal Club, June 2019.
\item Magellanic Cloud Fest III, University of Arizona, May, 2019.
\item JILA Seminar, JILA Institute, University of Colorado, December 2017.
\item STScI Galaxies Journal Club. December 2016.
\item Magellanic Cloud Fest II, University of Arizona, March 2016.
\item UNAM, Mexico City, Mexico, Nov 2013. 
\item Centro de Investigaciones De Astronom\'ia CIDA, Merida, Venezuela, July 2013.
\end{itemize}
\indent \dag \textit{Invited}

\section*{Posters}
\begin{itemize}
\setlength\itemsep{0.0em}
\renewcommand\labelitemi{$\cdot$}
  \item European Astronomical Society meeting, virtual meeting, June 2020.
  \item Rediscovering our Galaxy, IAU symposium 334, Potsdam, Germany. July 2017.
\end{itemize}

\section*{Telescope and HPC time Awarded}
\begin{itemize}
  \setlength\itemsep{0.0em}
  \renewcommand\labelitemi{$\cdot$}
\item Hubble Space Telescope, 32 orbits, Cycle 31, 2023. PI: Prof. Sukanya Chakrabarti (U. Alabama). 
\item MareNostrum Super computer, 1.7 million CPU hours, AECT-2023-2-0016, 2023.
  PI: Prof. Chervin Laporte (U Barcelona). 
\item GMRT 60 hours cycle 44, 2023. PI: Prof. Karin Menendez-Delmestre (Valongo Observatory, Rio de Janeiro).
\item Blanco Telescope, ``\textit{A VISTA-DECam Experiment in Near-Field
  Cosmology: Search for the Magellanic Dark Matter Wake}" cycle 2020B. PI: Prof. Julio
  Chaname (Universidad Catolica de Chile), 3 nights.
\end{itemize}


\section*{Grants}
\begin{itemize}
\setlength\itemsep{0.0em}
\renewcommand\labelitemi{$\cdot$}
\item LSSTC Grant Award (Virtual Internship in Rubin/LSST Science to Provide Research Experience to Undergraduate Students in Colombian Institutions)
  \href{https://lsstdiscoveryalliance.org/programs/science-catalyst-grants/2021/}{2021-51},
  2021. CO-PI: Nicol\'as Garavito-Camargo. This grant was awarded to support undergraduate research summer projects. 
\item International Astronomical Union (IAU) Office of Astronomy development
  (OAD)
  \href{https://www.astro4dev.org/category/a-virtual-community-mentorship-program-for-development-in-colombia/}{grant
  2021}. PI: Nicol\'as Garavito-Camargo. This grant was awarded to support the \href{https://www.astroreca.org/en/mentoring}{mentorship program for undergraduate students} in Colombia. 
\item University of Arizona theory travel grant, 2016 and 2018. 
\item IAU travel grant, 2016.
\end{itemize}

\section*{Observing Experience}
\begin{itemize}
  \setlength\itemsep{0.0em}
  \renewcommand\labelitemi{$\cdot$}
    \item DECam, Blanco-4m telescope, CTIO, Chile, 3 nights, 2020.
    \item VATT telescope. Mt Graham, Arizona, 4 nights, 2016.
    \item CIDA, Merida, Venezuela - 2 nights, 2013.
\end{itemize}
%%%%%%%%%%%%%%  Professional Experience %%%%%%%%%%%%%%%%%%5


\section*{Research highlights in the news}

%ADD DATES *** authors and details and links
\begin{itemize}
  \setlength\itemsep{0.0em}
  \renewcommand\labelitemi{$\cdot$}
  \item Sky \& Telescope: How our largest dwarf galaxy keeps other in line
  \item JPL Nasa: Astronomers Release New All-Sky Map of Milky Way’s Outer Reaches
  \item Syfy Wire: Dark Matter could be powering a galaxy that orbits the Milky Way until they collide
  \item Astrobites: 
  \item Phys.org: Astronomers release new all-sky map of the Milky Way's outer reaches
 \item University of Arizona news: Astrophysicist help chart dark matters invisible ocean
 \item AAS NOVA 2020: An Asymmetric Dark Matter Halo
 \item AAS NOVA 2019: Hunting for a Dark Matter Wake

\end{itemize}

%%%%%%%%%%%%%%% Publications %%%%%%%%%%%%%%%%%%%%%

%## Group first author and second authors paper were I have significant %contributions (Emily's and Hayden's as a student lead paper). Also students.


\section*{Publications list}

Refereed: 33-- First author: 5 -- Supervised students: 3 (denoted by \dag ) --
Co-supervised students (denoted by \ddag) $h-$index: 15 -- citations : 1233 (as of May 7th, 2025) 

\noindent \href{https://orcid.org/0000-0001-7107-1744}{ORCID},
\href{https://ui.adsabs.harvard.edu/search/q=docs(library%2F0X5_bcuLT4iE-6-Nko0kmg)&sort=date%20desc%2C%20bibcode%20desc&p_=0}{ADS},
\href{https://arxiv.org/search/?query=garavito-camargo&searchtype=all}{arXiv},
\href{https://scholar.google.com/citations?user=QDLiOFYAAAAJ&hl=en&oi=ao}{Google
Scholar}\\

\subsubsection*{First author publications or with significant contributions as mentor}


\begin{etaremune}
  \setcounter{enumi}{34}
\item \dag \textit{Shaping the Milky Way: The interplay of mergers and cosmic
  filaments }.
  Arpit Arora, \textbf{Nicol\'as
  Garavito-Camargo},  Robyn E. Sanderson, Martin D. Weinberg, Michael S. Petersen, Silvio Varela-Lavin, Facundo A. Gómez, Kathryn V. Johnston, Chervin F. P. Laporte, Nora Shipp, Jason A. S. Hunt, Gurtina Besla, Elise Darragh-Ford, Nondh Panithanpaisal, Kathryne J. Daniel. (ApJ Submitted 2025).

\item \textit{Implications for a High Mass M31: M33’s Orbital History and M31’s
  Response to the Passage of M33}. Ekta Patel, \textbf{Nicol\'as
  Garavito-Camargo}, Ivanna Escala. (ApJ in press 2025.)

\item \dag \textit{LMC Calls, Milky Way Halo Answers: Disentangling the Effects of
  the MW--LMC Interaction on Stellar Stream Populations}  Richard A. N.
  Brooks, \textbf{Nicol\'as Garavito-Camargo}, Kathryn V. Johnston, Adrian M. Price-Whelan,
  Jason L. Sanders, Sophia Lilleengen. ApJ in press 2024. [2 citations]

\item \ddag \textit{Segue 2 Recently Collided with the Cetus-Palca Stream: New
    Opportunities to Constrain Dark Matter in an Ultra-Faint Dwarf} Hadyen R.
    Foote, Gurtina Besla, \textbf{Nicol\'as Garavito-Camargo}, Ekta Patel, Guillaume F.
    Thomas, Ana Bonaca, Adrian M. Price-Whelan, Annika H. G. Peter, Dennis
    Zaritsky, Charlie Conroy. ApJ in press 2024. 

\item \dag \textit{LMC-driven anisotropic boosts in stream–subhalo
  interactions.} Arora, A., \textbf{Garavito-Camargo, N.}, Sanderson, R. E.,
  Cunningham, E. C., Wetzel, A., Panithanpaisal, N., Barry, M. ApJ 974, 2, 2024.
  [8 citations]

\item \textit{On the co-rotation of Milky Way satellites: LMC-mass satellites induce apparent motions in outer halo tracers}  
  \textbf{Nicol\'as Garavito-Camargo}, Adrian M. Price-Whelan , Emily C. Cunningham, Jenna
Samuel, Ekta Patel, Andrew Wetzel, Kathryn V. Johnston, Arpit Arora, Robyn E.
Sanderson, Lehman Garrison, and Danny Horta. ApJ, 975, 1, 2023. [8 citations] 

\item \textit{The Clustering of Orbital Poles Induced by the LMC: Hints for
      the Origin of Planes of Satellites}
      \textbf{Garavito-Camargo, Nicol\'as}; Patel, Ekta; Besla, Gurtina; Price-Whelan,
      Adrian; Laporte, Chervin; G\'omez, Facundo A; Kathryn V. Johnston; ApJ
      923, 2, 2021. [30 citations]

\item \textit{Quantifying the impact of the Large Magellanic Cloud on the
  structure of the Milky Way’s dark matter halo using Basis Function Expansions} \textbf{Garavito-Camargo, Nicol\'as}; Besla, Gurtina; Laporte,
  Chervin; Price-Whelan, Adrian M.; et al. ApJ, 919, 109, (2021). [95 citations] 

\item \textit{Quantifying the Stellar Halo's Response to the LMC's Infall with
  Spherical Harmonics}. Cunningham, Emily C; \textbf{Garavito-Camargo, Nicolas}, Deason, Alis J;
  Johnston, Kathryn V. et al. ApJ, 898, 1,(2020). [52 citations]


\item \textit{Hunting for the DM Wake induced by the LMC}.
  \textbf{Garavito-Camargo, Nicolas}; Besla, Gurtina; Laporte, Chervin F.P;
  Johnston, Kathryn V; G\'omez, Facundo A; Watkins, Laura. ApJ Accepted, (2019).
  [162 citations]

 \item \textit{The impact of gas bulk rotation on the morphology of the
   Lyman-alpha line}.\textbf{Garavito-Camargo, J.N}; Forero-Romero J.E;
   Dijkstra M. ApJ, 795, 120, (2014). [10 citations]

\end{etaremune}


\subsubsection*{Publications with significant contributions}

\begin{etaremune}
  \setcounter{enumi}{23}
  
  \item \textit{The distant Milky Way halo from the Southern hemisphere:
    Characterization of the LMC-induced dynamical-friction wake}. Manuel
    Cavieres, Julio Chanam\'e, Camila Navarrete, Yasna Ordenes-Brice\~no,
    \textbf{Nicol\'as Garavito-Camargo}, Gurtina Besla, Maren Hempel, Katherina Vivas, Facundo
    G\'omez (ApJ in press 2024).

  \item \textit{The All-Sky Impact of the LMC on the Milky Way Circumgalactic
  Medium}\\ Christopher Carr, Gerg L. Bryan, \textbf{Nicol\'as Garavito}, Gurtina Besla,
  David J. Setton, Kathryn V. Johnston (ApJ submitted 2024).

\item \textit{All-Sky Kinematics of the Distant Halo: The Reflex Response to the
  LMC}\\ Vedant Chandra, Rohan P. Naidu, Charlie Conroy, \textbf{Nicolas
  Garavito-Camargo}, Chervin F. P. Laporte, Ana Bonaca et al. (Submitted to ApJ 2024).


\item \textit{Structure, Kinematics, and Observability of the Large Magellanic
  Cloud's Dynamical Friction Wake in Cold vs. Fuzzy Dark Matter}\\
 Hayden R. Foote, Gurtina Besla, Philip Mocz, \textbf{Nicol\'as
 Garavito-Camargo},
 Lachlan Lancaster, Martin Sparre, Emily C. Cunningham, Mark Vogelsberger, Facundo A. Gómez , and Chervin F. P. Laporte, ApJ submitted 2023\\


\item \textit{Implications of the Milky Way travel velocity for dynamical mass
  estimates of the Local Group}\\
  Katie Chamberlain, Adrian M. Price-Whelan, Gurtina Besla, Emily C. Cunningham, \textbf{Nicol\'as Garavito-Camargo}, Jorge Peñarrubia, Michael S. Petersen. ApJ (2022)\\  


\item \textit{Detection of the All-Sky Response of the Galactic
  Halo to the Magellanic Clouds}\\ 
  Conroy, Charlie; Naidu, Rohan P; \textbf{Garavito-Camargo; Nicol\'as}; Besla,
  Gurtina; et al. Nature (2021). 


\item \textit{The orbital histories of Magellanic Satellites Using Gaia DR2
  proper motions}. \\
  Patel, Ekta; Kallivayalil, Nitya; \textbf{Garavito-Camargo, Nicolas} et. al.,
  ApJ 893, 121, (2020).


\item \textit{Modelling the gas kinematics of an atypical Ly$\alpha$
emitting compact dwarf galaxy.}\\  Forero-Romero, Jaime E., Gronke, Max.,
Remolina-Gutiérrez, Maria Camila, \textbf{Garavito-Camargo, Nicolas}, Dijkstra
M. MNRAS, 474, 12F, (2018).

\end{etaremune}


\subsubsection*{Publications with moderate contributions}

\begin{etaremune}
  \setcounter{enumi}{15}

\item \textit{Where do High-Velocity Dark Matter Particles come from in the
  Milky Way?}. Aidan DeBrae, Peter Behroozi, \textbf{Nicolas Garavito-Camargo},
  Submitted to OJA 2025.

\item \textit{Hypervelocity Stars Trace a Supermassive Black Hole in the Large
  Magellanic Cloud}. Jiwon Jesse Han, Kareem El-Badry, Scott Lucchini, Lars
  Hernquist, Warren Brown, Nico Garavito-Camargo, Charlie Conroy, Re'em Sari.
  ApJ in press 2025. 

\item \textit{Efficient and accurate force replay in cosmological-baryonic
  simulations}. Arpit Arora, Robyn Sanderson, Christopher Regan, \textbf{Nicol\'as
  Garavito-Camargo}, Emily Bregou, Nondh Panithanpaisal, Andrew Wetzel, Emily
  Cunningham, Sarah Loebman, Adriana Dropulic, Nora Shipp. ApJ 977, 1, 2024. 


\item \textit{Generating synthetic star catalogs from simulated data for next-gen observatories with py-ananke}\\
  Adrien C. R. Thob, Robyn E. Sanderson, Andrew P. Eden, Farnik
  Nikakhtar, Nondh Panithanpaisal, \textbf{Nicol\'as Garavito-Camargo},
  and Sanjib Sharma (In press JOSS 2024).

\item \textit{Dark matter distribution in Milky Way-analog galaxies}\\ 
 Natanael Gomes-Oliveira, K. Menéndez-Delmestre, T. S. Gonçalves, D. C.
 Rodrigues, M. Grossi, \textbf{N. Garavito-Camargo}, A. Araújo, P. P. B. Beaklini, Y.
 Cavalcante-Coelho, A. Cortesi, L. H. Quiroga-Nuñez, T. Randriamampandry. ApJ 2023.

\item \textit{The proto-galaxy of Milky Way-mass haloes in the FIRE simulation}
  \\ Horta, Danny; Cunningham, Emily C.; Sanderson, Robyn; Johnston, Kathryn V.;
  Deason, Alis; Wetzel, Andrew; McCluskey, Fiona; \textbf{Garavito-Camargo,
  Nicol\'as}; Necib, Lina; Faucher-Giguère, 
  Claude-André ; Arora, Arpit ; Gandhi, Pratik J. (ApJ Submitted 2023)\\

\item \textit{Galactoseismology in cosmological simulations: Vertical
  perturbations by dark matter, satellite galaxies and gas.}  Garcia-Conde. B,
  Antoja. T, Roca-Fabrega. S, G\'omez. G, Ramos. P,. \textbf{Garavito-Camargo.
  N}, G\'omez-Flechoso,  MA. (accepted for publication in MNRAS 2023)\\ 


\item \textit{The impact of the Large Magellanic Cloud on dark matter direct detection signals}\\
 Smith-Orlik, Adam ; Ronaghi, Nima ; Bozorgnia, Nassim ; Cautun, Marius ; Fattahi, Azadeh ; Besla, Gurtina ; Frenk, Carlos S. ; \textbf{Garavito-Camargo, Nicol\'as} ; Gómez, Facundo A. ; Grand, Robert J. J. ; Marinacci, Federico ; Peter, Annika H. G. JCAP submitted 2023\\

\item \textit{Lopsided Galaxies in a cosmological context: a new galaxy-halo connection}\\ 
Silvio Varela-Lavin, Facundo A. Gómez, Patricia B. Tissera, Gurtina Besla, \textbf{Nicol\'as, Garavito-Camargo}, Federico Marinacci. Submitted to MNRAS 2022.\\ 

\item \textit{On the stability of tidal streams in action space}\\
  Arpit Arora, Robyn E. Sanderson, Nondh Panithanpaisal, Emily C. Cunningham, Andrew Wetzel, \textbf{Nicol\'as Garavito-Camargo}. ApJ l, vol. 939, no. 1, (2022). 

\item \textit{The highest-speed local dark matter particles come from the Large
  Magellanic Cloud}. \\
  Besla, Gurtina; Peter, Annika; \textbf{Garavito-Camargo, Nicolas}. JCAP, 11,
  13, (2019).

\item \textit{The influence of Sagittarius and the Large Magellanic Cloud on the
  stellar disc of the Milky Way Galaxy.}\\
  Laporte, Chervin F. P; Johnston, Kathryn V; G\'omez, Facundo A; \textbf{Garavito-Camargo, Nicolas}; Besla,
  Gurtina. MNRAS, 481, 286L, (2018).

\item \textit{The Extremely Luminous Quasar Survey in the Sloan Digital Sky
  Survey Footprint. II. The North Galactic Cap Sample.}\\ Schindler, Jan-Torge;
  Fan, Xiaohui; McGreer, Ian D; Yang, Jinyi; Wang, Feige; Green, Richard;
  \textbf{Garavito-Camargo, Nicolas} et al., ApJ, 863, 144S, (2018).

\item \textit{Response of the Milky Way's disc to the Large Magellanic Cloud in
  a first infall scenario.}\\ Laporte, C.; Gomez, F; Besla, Gurtina; Johnston,
  Kathryn V; \& \textbf{Garavito-Camargo, Nicolas}. MNRAS, 473, 1218L, (2018).

\end{etaremune}




\section*{White papers}

\begin{etaremune}
  \setcounter{enumi}{4}
\item \textit{NANCY: Next-generation All-sky Near-infrared Community surveY}
  Jiwon Jesse Han, et al. (incl. \textbf{Garavito-Camargo,
  N}).\href{https://arxiv.org/abs/2306.11784}{call for white papers for the
  Roman Core Community Survey}
\item \textit{Mass Spectroscopy of the Milky Way} \\
  Dey, Arjun. et al., (incl. \textbf{Garavito-Camargo, N}).
  \href{https://113qx216in8z1kdeyi404hgf-wpengine.netdna-ssl.com/wp-content/uploads/2019/05/489_dey.pdf}{Astro2020: Decadal
  Survey on Astronomy and Astrophysics, Vol. 51, Issue 3, id. 489 (2019).}
\item \textit{The Multidimensional Milky Way.}\\
 Sanderson, Robyn E .et al. (incl. \textbf{Garavito-Camargo}) 2019.
  \href{https://arxiv.org/abs/1909.07641}{Astro2020: Decadal Survey on
Astronomy and Astrophysics, 2019, Vol. 51, Issue 3, id. 347 (2019).} 
\end{etaremune}


\end{document}



