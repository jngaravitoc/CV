\documentclass[UTF8]{article}
\usepackage{enumerate}
\usepackage[hmargin=2cm,vmargin=2.5cm]{geometry}
\usepackage[colorlinks=true]{hyperref}
\usepackage{url}
\usepackage{etaremune}
\usepackage{fancyhdr}
\usepackage{lastpage}
\usepackage[dvipsnames]{xcolor}
\usepackage{sectsty}
\usepackage[dvipsnames]{xcolor}


\hypersetup{
  colorlinks,
  linkcolor={red!50!black},
  citecolor={blue!50!black},
  urlcolor={SeaGreen!90!black}}


\pagestyle{fancy}
\fancyhf{}
\lhead{Nicol\'as Garavito-Camargo}
\chead{Curriculum Vitae}
\rhead{\thepage}
\thispagestyle{empty}
%sectionfont{\color{}}

\fancyfoot[C]{\textit{Last updated: \today}}

\begin{document}
\begin{center}
\indent \textbf{\LARGE Nicol\'as Garavito-Camargo -- Curriculum Vitae} \\
\indent \rule{17cm}{0.4pt}\\
\end{center}

\begin{center}
  Flatiron Research Fellow $|$ \href{https://www.simonsfoundation.org/flatiron/center-for-computational-astrophysics/} {Center for Computational Astrophysics
(CCA)}, NYC\\
E-mail: ngaravito@flatironinstitute.org $|$ Website:
\href{http://jngaravitoc.github.io/Garavito-Camargo}{jngaravitoc.github.io} $|$ Github: \href{http://www.github.com/jngaravitoc}{github.com/jngaravitoc}\\
\end{center}

%Phone: +1(520)265-6092
%\section*{Research Interests}
%Galaxy Dynamics, Local Group Dynamics, Dwarf Galaxies, Galactic
%Archaeology, Astrophysical probes of Dark Matter, Galaxy Formation and Evolution.



\section*{Education}
\textbf{Ph.D.,} Astronomy and Astrophysics, \href{https://www.as.arizona.edu/}{University of Arizona}, 2021.\\
\indent \textit{Advisor: \href{https://sites.google.com/view/thebeslagroup/home}{Dr. Gurtina Besla}}\\
\textbf{M.Sc.,}  Physics, \href{https://fisica.uniandes.edu.co/en}{Universidad de Los Andes}, Bogot\'a, Colombia, 2015.\\
\indent \textit{Advisor: \href{http://wwwprof.uniandes.edu.co/~je.forero/}{Dr. Jaime E. Forero-Romero}}\\
\textbf{B.Sc.,} Physics,
\href{https://unal.edu.co/en.html}{Universidad Nacional de Colombia}, Bogot\'a, Colombia, 2013.\\
\indent \textit{Advisor:
\href{https://www.iau.org/administration/membership/individual/16146/}{Dr.
Rigoberto A. Casas Miranda}}


\section*{Appointment}
\indent Flatiron Research Fellow, Flatiron Institute, October 2021-September 2024.

\section*{Scholarships and Awards}


\begin{itemize}
  \setlength\itemsep{0.0em}
  \renewcommand\labelitemi{$\cdot$}

\item University of Arizona, Theoretical Astrophysics Program, Graduate Student
Research Prize 2021.
\item University of Arizona College of Science award for Excellence in
Service for graduate students, 2020.\\ 
\indent \textit{(Awarded to one graduate student across the college of
science per year.)}
\item University of Arizona theory travel grant, 2016, 2018. 
\item McCarthy-Stoeger Scholarship 2015-2017, Vatican Observatory.
\end{itemize}
%SpaceArt scholarship, Bogot\'a Planetarium: For developing art and science material for children, 2014. \\
%Best research project at the Gaia Summer school, UNAM, Mexico City, 2013.
%2010 - $1^{st}$ Place Maloka (interactive Science Museum) contest to design a didactical physics experiment.\\


\section*{Student research project supervision}

Total students advised or co-advised 6; 3 Ph.D, 3 undergraduates.

\begin{itemize}
  \setlength\itemsep{0.0em}
  \renewcommand\labelitemi{$\cdot$}
  \item Silvio Varela (Graduate student at Universdad de la Serena); Fall 2022-present. I am
    currently co-advising Silvio in one research project. 
  \item Arpit Arora (Graduate student at University of Pennsylvania); Fall 2021-present. I am
    currently co-adivising Arpit Arora in two research projects.
  \item Hayden Foote (Graduate student at University of Arizona); Fall 2021 - present. I co-advised Hayden in a research project that
  resulted in a submitted publication.    
  \item Andrew Eden (Undergraduate at Florida Institute of Technology); Fall 2022-present.
    I am currently advising Andrew on his undergraduate thesis project. 
  \item Ludia Adhikary (Undegraduate CUNY); co-supervised with Emily Cunningham. Through the AstroCOM CUNY/CCA program; NYC Summer 2022-present.
\item Stephanie Carolina Cely Rodriguez (Undergraduate at Universidad Nacional de Colombia);
  co-advising undergraduate thesis; Summer 2022-present.
\end{itemize}

\section*{Teaching Experience}

\begin{itemize}
  \setlength\itemsep{0.0em}
  \renewcommand\labelitemi{$\cdot$}
\item Guest Lecturer for the graduate Galaxies class, Columbia University, Fall 2022.
\item TA, Astronomy Tutoring for Majors \& Minors Program. University of Arizona, Spring 2019. 
\item Computational Physics, Teaching assistant, PHYS305. University of Arizona. Spring 2018.
\item Computational Tools: Universidad de los Andes, $1^{st}$ 2014-2015.
\item Computational Methods Laboratory: Universidad de los Andes, $1^{st}$ 2014-2015.
\item TA, Computational Methods: Universidad de los Andes, $2^{nd}$ 2014.
\item Physics I Laboratory: Universidad de los Andes, $2^{nd}$ Semester-2013.
\end{itemize}


\subsection*{Academic Service:} 

\begin{itemize}
  \setlength\itemsep{0.0em}
  \renewcommand\labelitemi{$\cdot$}
\item Referee of MNRAS, Nature, ApJ, Galaxies Journal. 2018-present.
\item B-BFE coordinator: I lead the cosmologial simulation group, organize in person and online meetings.
\item Mentor at the CUNY-CCA program for undergraduate students at CUNY working with mentors at the CCA, summer 2022. 
\item Proposal reviewer for \href{https://clubesdeciencia.co/}{Colombian Science Clubs}, 2018.
\item Local Organizing Committee of the \href{http://forero.github.io/AndeanCosmologySchool/}{Andean Cosmology School}, Universidad de Los Andes, 2015.
\item Organizer of the student astronomy seminar at the Planetarium of Bogot\'a, 2014-2015.
\end{itemize}

\section*{Scientific Talks:}

44 Total: 17 Invited (denoted by \dag), 26 in North America, 6 in Europe, 9 in Latin America, 1 Asia.\\


\subsection*{Conferences (13)}



\begin{itemize}
  \setlength\itemsep{0.0em}
  \renewcommand\labelitemi{$\cdot$}

\item Friends of Friends meeting, Cordoba, Argentina, April, 2023. \dag
\item IAU 379: Dynamical masses of local group galaxies, contributed talk, March, 2023.
\item \href{https://www.mso.anu.edu.au/~yting/Malaysia_IAU/}{IAU 377: Early Disk-Galaxy Formation
from JWST to the Milky Way}, contributed talk, February, 2023.
\item \href{https://accefyn.com/microsites/nodos/astroco/congreso-colombiano-de-astronomia-cocoa-2022%E2%80%8B/}{Colombian Congress of Astronomy}, Plenary talk, August, 2022. \dag
\item \href{http://fof.oac.uncor.edu/2022/}{Friends of friends meeting}, Cordoba, Argentina, April, 2022.
\item \href{https://aas.org/meetings/dda52}{Division on Dynamics Astronomy}, Virtual meeting, May 2021.
\item \href{https://stellarstreams.org/streams21/}{Streams 21}, Virtual
  Conference, February 2021. 
\item  \href{https://www.stsci.edu/contents/events/stsci/2020/april/the-local-group-assembly-and-evolution?page=2&filterUUID=6fedb8a7-}{The Local Group: Assembly and Evolution}, virtual conference, August 2020.
\item \href{https://eas.unige.ch/EAS2020/}{European Astronomical Society meeting}, virtual meeting, June 2020.
\item Durham University, UK, \href{http://astro.dur.ac.uk/cosmodwarfs/}{Small Galaxies Cosmic Questions}, August, 2019.
\item MPIA, Heidelberg, \href{http://www.mpia.de/homes/stellarhalos2018-loc/sh2018/index.html}{Stellar halos across the cosmos}, July 2018.
\item LARIM, Cartagena, Colombia, October 2016.
\item EWASS, Geneve, Switzerland, July 2014.
\end{itemize}
  
\subsection*{Seminars (31)}
\begin{itemize}
  \setlength\itemsep{0.0em}
  \renewcommand\labelitemi{$\cdot$}
\item Universidad de La Serena, La Serena, Chile, May 2023. \dag. 
\item Instituto de Astronom\'ia y F\'isica del Espacio (IAFE), Buenos Aires, April, 2023.
\item Max Planck Institute for Astrophysics, Cosmology Seminar, March, 2023.
\item U. Rutgers, Astronomy Seminar, Nov, 2022. \dag
\item U. Columbia, Lunch talk, Sept, 2022. \dag
\item STScI, galaxies lunch talk, May, 2022. \dag
\item Universidad de Ant\'ioquia, seminar, February, 2022. \dag 
\item University of Massachusetts, Amherst, Colloquium, Jan, 2022. \dag
\item University of Madison-Wisconsin, Science seminar, November 2021.\dag
\item University of Michigan, Galaxies group seminar, November 2021.\dag
\item NYU, CCPP, seminar, November 2021.\dag 
\item CCA, Flatiron Institute, Lunch Talk, October, 2021.
\item Steward Observatory, Theoretical Astrophysics Program (TAP) colloquium, September, 2021.\dag
\item \href{https://comscicon.com/comscicon-en-espa%C3%B1ol-2021}{ComSciCon},
    June 2021. \dag. 
\item Steward Observatory, Galaxy lunch talk, March, 2021.
\item Univesidad de los Andes, Astronomy Seminar, February, 2021. 
\item UC Irvine, Astronomy Seminar, January, 2021. \dag
\item Steward Observatory, Early Career Scientist talk, December, 2020. \dag
\item CCAPP, Seminar, December, 2020. \dag
\item Princeton, Journal Club, November 2020. \dag
\item KIPAC, Stanford, Tea-Talk, October 2020.
\item Harvard Center for Astrophysics, GCSP seminar, October 2020.
\item Carnegie Observatories, seminar, October 2020. \dag
\item University of California Berkeley, lunch talk, September 2020. 
\item The Royal Observatory of Edinburgh, UK, Coffee Talk, August 2019. \dag
\item W.M. Keck Observatory, Journal Club, June 2019.
\item Magellanic Cloud Fest III, University of Arizona, May, 2019.
\item JILA Seminar, JILA Institute, University of Colorado, December 2017.
\item STScI Galaxies Journal Club. December 2016.
\item Magellanic Cloud Fest II, University of Arizona, March 2016.
\item UNAM, Mexico City, Mexico, Nov 2013. 
\item Centro de Investigaciones De Astronom\'ia CIDA, Merida, Venezuela, July 2013.
\end{itemize}
\indent \dag \textit{Invited}

\section*{Posters}
\begin{itemize}
\setlength\itemsep{0.0em}
\renewcommand\labelitemi{$\cdot$}
  \item European Astronomical Society meeting, virtual meeting, June 2020.
  \item Rediscovering our Galaxy, IAU symposium 334, Potsdam, Germany. July 2017.
\end{itemize}

\section*{Telescope Time Awarded}
\begin{itemize}
  \setlength\itemsep{0.0em}
  \renewcommand\labelitemi{$\cdot$}
\item Blanco Telescope, ``\textit{ A VISTA-DECam Experiment in Near-Field
  Cosmology: Search for the Magellanic Dark Matter Wake}" cycle 2020B. PI: Julio Chaname, 3 nights
  \item GRTM 40 hours cycle 44. 
\end{itemize}


\section*{Grants}
\begin{itemize}
\setlength\itemsep{0.0em}
\renewcommand\labelitemi{$\cdot$}
\item LSSTC Grant Award \#2021-51, 2001. PI: Andres A. Plazas Malag\'on
\item IAU OAD grand.
\item University of Arizona theory travel grant, 2016, 2018. 
\item IAU travel grant.2016.
\end{itemize}

\section*{Observing Experience}
\begin{itemize}
  \setlength\itemsep{0.0em}
  \renewcommand\labelitemi{$\cdot$}
    \item DECam, Blanco-4m telescope, CTIO, Chile. 3 nights, 2020.
    \item VATT telescope. Mt Graham, Arizona - 4 nights, 2016.
    \item CIDA, Merida, Venezuela - 2 nights, 2013.
\end{itemize}
%%%%%%%%%%%%%%  Professional Experience %%%%%%%%%%%%%%%%%%5

\section*{DEI and outreach:}


\subsection*{DEI Leadership:}
\begin{itemize}
  \setlength\itemsep{0.0em}
  \renewcommand\labelitemi{$\cdot$}
\item Co-creator of the 10-week
  \href{https://recastronomia.github.io/internship/}{RECA internship program}
  for astronomy students in Colombia. May-August 2021.
\item Co-creator of the \href{https://recastronomia.github.io/mentores/}{RECA mentorship program} for astronomy students in Colombia. 2020-present.
  (The mentorship program pairs up students with professional astronomers to
  provide guidance through the application process to graduate programs)
\item Co-organizer,
  \href{https://www.as.arizona.edu/diversity_coffee/}{Diversity Journal Club} Steward Observatory, 2018-2021.
\item Creator, \href{https://astrocharlas.github.io/}{Astrocharlas},
Steward Observatory, 2018-present.
 (Spanish outreach series talks in astronomy)
\item Writer for \href{https://astrobitos.org/}{Astrobitos}, 2018-present.
\item Mentor, Tucson Initiative for Minority Engagement in Science and Technology Program \href{https://lavinia.as.arizona.edu/~timestep/}{TIMESTEP}, 2016-2018.
\end{itemize}

\subsection*{Outreach}
\begin{itemize}
  \setlength\itemsep{0.0em}
  \renewcommand\labelitemi{$\cdot$}
\item Classroom astronomer, NOAO Project \href{https://www.noao.edu/education/astro/}{ASTRO}, 2018-2019.
\item Discussion leader, NOAO \href{http://www.teenastronomycafe.org/}{Teen Astronomy Cafe}, 2018.
\item Planetarium SpaceArt mentor for Children, Bogot\'a, Colombia, 2013-2014.
%\indent\indent Unconscious Bias Workshop, 2017. \\
\end{itemize}





\section*{Research highlights in the news}

%ADD DATES *** authors and details and links
\begin{itemize}
  \setlength\itemsep{0.0em}
  \renewcommand\labelitemi{$\cdot$}
  \item Sky \& Telescope: How our largest dwarf galaxy keeps other in line
  \item JPL Nasa: Astronomers Release New All-Sky Map of Milky Way’s Outer Reaches
  \item Syfy Wire: Dark Matter could be powering a galaxy that orbits the Milky Way until they collide
  \item Astrobites: 
  \item Phys.org: Astronomers release new all-sky map of the Milky Way's outer reaches
 \item University of Arizona news: Astrophysicist help chart dark matters invisible ocean
 \item AAS NOVA 2020: An Asymmetric Dark Matter Halo
 \item AAS NOVA 2019: Hunting for a Dark Matter Wake

\end{itemize}

%%%%%%%%%%%%%%% Publications %%%%%%%%%%%%%%%%%%%%%

%## Group first author and second authors paper were I have significant contributions (Emily's and Hayden's as a studen lead paper). Also students.

%Papers in prep -- draft availblae upon request for the orbital poles paper.

\section*{Publications list}



\noindent \href{https://orcid.org/0000-0001-7107-1744}{ORCID},
\href{https://ui.adsabs.harvard.edu/search/q=docs(library%2F0X5_bcuLT4iE-6-Nko0kmg)&sort=date%20desc%2C%20bibcode%20desc&p_=0}{ADS},
\href{https://arxiv.org/search/?query=garavito-camargo&searchtype=all}{arXiv}\\
refereed: 21 -- first author: 5  -- $h-$index: 11 -- citations : 765 (as of Sept 29th/2023) 
\begin{etaremune}
  \setcounter{enumi}{22}


\item \textit{LMC-driven anisotropic boosts in stream–subhalo interactions} \\
  Arora, A., \textbf{Garavito-Camargo, N.}, Sanderson, R. E., Cunningham, E. C., Wetzel, A., Panithanpaisal, N., Barry, M. (ApJ submitted 2023).

\item \textit{The proto-galaxy of Milky Way-mass haloes in the FIRE simulation}
  \\ Horta, Danny; Cunningham, Emily C.; Sanderson, Robyn; Johnston, Kathryn V.;
  Deason, Alis; Wetzel, Andrew; McCluskey, Fiona; Garavito-Camargo, Nicol\'as ; Necib, Lina; Faucher-Giguère, 
  Claude-André ; Arora, Arpit ; Gandhi, Pratik J. (ApJ Submitted 2023)\\

\item \textit{Galactoseismology in cosmological simulations: Vertical
  perturbations by dark matter, satellite galaxies and gas.}\\ Garcia-Conde. B,
  Antoja. T, Roca-Fabrega. S, G\'omez. G, Ramos. P,. Garavito-Camargo. N, G\'omez-Flechoso,
  MA. (submitted 2023)\\ 

\item \textit{LMC-mass satellite galaxies inducing ‘apparent’ co-rotation motions in Milky Way-like galaxies in
FIRE}\\ 
Nicol\'as Garavito-Camargo, Adrian M. Price-Whelan , Emily C. Cunningham, Jenna
Samuel, Ekta Patel, Andrew Wetzel, Kathryn V. Johnston, Arpit Arora, Robyn E.
Sanderson, and Lehman Garrison (submitted 2023. Draft available upon request.) 


\item \textit{Structure, Kinematics, and Observability of the Large Magellanic
  Cloud's Dynamical Friction Wake in Cold vs. Fuzzy Dark Matter}\\
 Hayden R. Foote, Gurtina Besla, Philip Mocz, Nicol\'as Garavito-Camargo,
 Lachlan Lancaster, Martin Sparre, Emily C. Cunningham, Mark Vogelsberger, Facundo A. Gómez , and Chervin F. P. Laporte, ApJ submitted 2023\\


\item \textit{The impact of the Large Magellanic Cloud on dark matter direct detection signals}\\
 Smith-Orlik, Adam ; Ronaghi, Nima ; Bozorgnia, Nassim ; Cautun, Marius ; Fattahi, Azadeh ; Besla, Gurtina ; Frenk, Carlos S. ; \textbf{Garavito-Camargo, Nicol\'as} ; Gómez, Facundo A. ; Grand, Robert J. J. ; Marinacci, Federico ; Peter, Annika H. G. JCAP submitted 2023\\

\item \textit{Lopsided Galaxies in a cosmological context: a new galaxy-halo connection}\\ 
Silvio Varela-Lavin, Facundo A. Gómez, Patricia B. Tissera, Gurtina Besla, \textbf{Nicolás Garavito-Camargo}, Federico Marinacci. Submitted to MNRAS 2022.\\ 

\item \textit{On the stability of tidal streams in action space}\\
  Arpit Arora, Robyn E. Sanderson, Nondh Panithanpaisal, Emily C. Cunningham, Andrew Wetzel, \textbf{Nicol\'as Garavito-Camargo}. ApJ l, vol. 939, no. 1, (2022). 


\item \textit{Implications of the Milky Way travel velocity for dynamical mass
  estimates of the Local Group}\\
  Katie Chamberlain, Adrian M. Price-Whelan, Gurtina Besla, Emily C. Cunningham, \textbf{Nicol\'as Garavito-Camargo}, Jorge Peñarrubia, Michael S. Petersen. ApJ (2022)\\  

\item \textit{The Clustering of Orbital Poles Induced by the LMC: Hints for
      the Origin of Planes of Satellites}\\ 
      \textbf{Garavito-Camargo, Nicol\'as}; Patel, Ekta; Besla, Gurtina; Price-Whelan,
      Adrian; Laporte, Chervin; G\'omez, Facundo A; Kathryn V. Johnston; ApJ
      in press (2021). 


\item \textit{Detection of the All-Sky Response of the Galactic
  Halo to the Magellanic Clouds}\\ 
  Conroy, Charlie; Naidu, Rohan P; \textbf{Garavito-Camargo; Nicol\'as}; Besla,
  Gurtina; et al. Nature (2021). 

\item \textit{Quantifying the impact of the Large Magellanic Cloud on the
  structure of the Milky Way’s dark matter halo using Basis Function Expansions}\\ 
  \textbf{Garavito-Camargo, Nicol\'as}; Besla, Gurtina; Laporte,
  Chervin; Price-Whelan, Adrian M.; et al. ApJ, 919, 109, (2021). 

\item \textit{Quantifying the Stellar Halo's Response to the LMC's Infall with
  Spherical Harmonics}.\\
  Cunningham, Emily C; \textbf{Garavito-Camargo, Nicolas}, Deason, Alis J;
  Johnston, Kathryn V. et al. ApJ, 898, 1,(2020).

\item \textit{The orbital histories of Magellanic Satellites Using Gaia DR2
  proper motions}. \\
  Patel, Ekta; Kallivayalil, Nitya; \textbf{Garavito-Camargo, Nicolas} et. al.,
  ApJ 893, 121, (2020).
\item \textit{The highest-speed local dark matter particles come from the Large
  Magellanic Cloud}. \\
  Besla, Gurtina; Peter, Annika; \textbf{Garavito-Camargo, Nicolas}. JCAP, 11,
  13, (2019).
\item \textit{Hunting for the DM Wake induced by the LMC}.\\
  \textbf{Garavito-Camargo, Nicolas}; Besla, Gurtina; Laporte, Chervin F.P;
  Johnston, Kathryn V; G\'omez, Facundo A; Watkins, Laura. ApJ Accepted, (2019).
\item \textit{The influence of Sagittarius and the Large Magellanic Cloud on the
  stellar disc of the Milky Way Galaxy.}\\
  Laporte, Chervin F. P; Johnston, Kathryn V; G\'omez, Facundo A; \textbf{Garavito-Camargo, Nicolas}; Besla,
  Gurtina. MNRAS, 481, 286L, (2018).
\item \textit{The Extremely Luminous Quasar Survey in the Sloan Digital Sky
  Survey Footprint. II. The North Galactic Cap Sample.}\\ Schindler, Jan-Torge;
  Fan, Xiaohui; McGreer, Ian D; Yang, Jinyi; Wang, Feige; Green, Richard;
  \textbf{Garavito-Camargo, Nicolas} et al., ApJ, 863, 144S, (2018).
\item \textit{Modelling the gas kinematics of an atypical Ly$\alpha$
emitting compact dwarf galaxy.}\\  Forero-Romero, Jaime E., Gronke, Max.,
Remolina-Gutiérrez, Maria Camila, \textbf{Garavito-Camargo, Nicolas}, Dijkstra
M. MNRAS, 474, 12F, (2018).
\item \textit{Response of the Milky Way's disc to the Large Magellanic Cloud in
  a first infall scenario.}\\ Laporte, C.; Gomez, F; Besla, Gurtina; Johnston,
  Kathryn V; \& \textbf{Garavito-Camargo, Nicolas}. MNRAS, 473, 1218L, (2018).
 \item \textit{The impact of gas bulk rotation on the morphology of the
   Lyman-alpha line}.\\ \textbf{Garavito-Camargo, J.N}; Forero-Romero J.E;
   Dijkstra M. ApJ, 795, 120, (2014).
\end{etaremune}




\section*{White papers}

\begin{etaremune}
\item \textit{Mass Spectroscopy of the Milky Way} \\
  Dey, Arjun. et al. (incl. \textbf{Garavito-Camargo}).
  \href{https://113qx216in8z1kdeyi404hgf-wpengine.netdna-ssl.com/wp-content/uploads/2019/05/489_dey.pdf}{Astro2020: Decadal
  Survey on Astronomy and Astrophysics, Vol. 51, Issue 3, id. 489 (2019).}
\item \textit{The Multidimensional Milky Way.}\\
 Sanderson, Robyn E .et al. (incl. \textbf{Garavito-Camargo}) 2019.
  \href{https://arxiv.org/abs/1909.07641}{Astro2020: Decadal Survey on
Astronomy and Astrophysics, 2019, Vol. 51, Issue 3, id. 347 (2019).} 
\end{etaremune}


\end{document}



