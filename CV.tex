\documentclass[UTF8]{article}
\usepackage{enumerate}
\usepackage[hmargin=2cm,vmargin=2.5cm]{geometry}
\usepackage[colorlinks=true]{hyperref}
\usepackage{url}
\usepackage{etaremune}
\usepackage{fancyhdr}
\usepackage{lastpage}
\usepackage[dvipsname]{xcolor}

\renewcommand{\familydefault}{\sfdefault}

%\definecolor{navyblue}{rgb}{0, 0.19, 0.56}

\pagestyle{fancy}
\fancyhf{}
\lhead{Nicol\'as Garavito-Camargo}
\chead{Curriculum Vitae}
\rhead{\thepage}
\thispagestyle{empty}

\fancyfoot[C]{\textit{Last updated: \today}}

\begin{document}

\indent \textbf{\LARGE Nicol\'as Garavito-Camargo}\\
\indent \rule{17cm}{0.4pt}\\

\indent\indent Steward Observatory, University of Arizona, 933 North Cherry Av. Tucson, AZ.\\
\indent\indent E-mail: jngaravitoc@email.arizona.edu\\ 
\indent\indent Website: \href{http://jngaravitoc.github.io/Garavito-Camargo}{jngaravitoc.github.io}\\
\indent\indent Github: \href{http://www.github.com/jngaravitoc}{github.com/jngaravitoc} \\

{\Large Research Interests}\\

\indent\indent Galaxy Dynamics, Local Group Dynamics, Dwarf Galaxies, Galactic
Archeology, Galaxy Formation and evolution, 
\indent\indent Astrophysical probes of Dark
Matter.\\


{\Large Education}\\


\indent\indent \textbf{Ph.D.,} Astronomy and Astrophysics, University of
Arizona, \textit{Expected May 2021}.\\
\indent\indent\indent\indent \ \textit{Advisor: Dr. Gurtina Besla}\\

\indent\indent\textbf{M.Sc.,}  Physics, University de Los Andes, Bogot\'a, Colombia, 2015.\\
\indent\indent\indent\indent \ \textit{Advisor: Dr. Jaime E. Forero-Romero}\\

\indent\indent\textbf{B.Sc.,} \  Physics, Universidad Nacional de Colombia, Bogot\'a, Colombia, 2013.\\
\indent\indent\indent\indent \ \textit{Advisor: Dr. Rigoberto A. Casas Miranda}\\

{\Large{Fellowships, Grants, and Awards}}\\

\indent\indent University of Arizona College of Science award for excellence in service. 2020. \\
\indent\indent University of Arizona theory travel grant, 2018. \\
\indent\indent University of Arizona theory travel grant, 2016. \\
\indent\indent IAU travel grant, 2016.\\
\indent\indent McCarthy-Stoeger scholarship, 2015-2017.\\
\indent\indent SpaceArt scholarship, Bogot\'a Planetarium: For developing art and science material for children, 2014. \\
\indent\indent Best project at the GAIA summer school, UNAM, Mexico City, 2013.\\
%2010 - $1^{st}$ Place Maloka (interactive Science Museum) contest to design a didactical physics experiment.\\


{\Large Publications}\\

\indent \indent Refereed: 8 - first author: 2.

\begin{enumerate}
\item \textit{The orbital histories of Magellanic Satellites Using Gaia DR2
  proper motions}. Patel, Ekta; Kallivayalil, Nitya; \textbf{Garavito-Camargo,
    Nicolas} et al. ApJ submitted, 2020.

\item \textit{The highest-speed local dark matter particles come from the Large
  Magellanic Cloud}. Bestla, Gurtina; Peter, Annika; \textbf{Garavito-Camargo,Nicolas}. Submitted to JCAP, 2019.
\item \textit{Hunting for the DM Wake induced by the LMC}.
  \textbf{Garavito-Camargo Nicolas}; Besla, Gurtina; Laporte, Chervin F.P;
    Johnston, Kathryn V; G\'omez, Facundo A; Watkins, Laura. ApJ Accepted, 2019.
\item \textit{The influence of Sagittarius and the Large Magellanic Cloud on the
  stellar disc of the Milky Way Galaxy.} Laporte, Chervin F. P; Johnston,
  Kathryn V; G\'omez, Facundo A; \textbf{Garavito-Camargo, Nicolas}; Besla,
    Gurtina. MNRAS, 481, 286L, 2018.
\item \textit{The Extremely Luminous Quasar Survey in the Sloan Digital Sky Survey Footprint. II. The North Galactic Cap Sample.} Schindler, Jan-Torge., Fan, Xiaohui., McGreer Ian D., Yang Jinyi., Wang, Feige., Green, Richard., \textbf{Garavito-Camargo, Nicolas} et al., ApJ, 863, 144S, 2018.
\item \textit{Modelling the gas kinematics of an atypical Ly$\alpha$
emitting compact dwarf galaxy.} Forero-Romero, Jaime E., Gronke, Max., Remolina-Gutiérrez, Maria Camila, \textbf{Garavito-Camargo, Nicolas}, Dijkstra M. MNRAS, 474, 12F, 2018.
\item \textit{Response of the Milky Way's disc to the Large Magellanic Cloud in
  a first infall scenario.} Laporte, C.; Gomez, F; Besla,Gurtina; Kathryn V. Johnston \& \textbf{Garavito-Camargo, Nicolas}. MNRAS, 473, 1218L, 2018.
 \item \textit{The impact of gas bulk rotation on the morphology of the
   Lyman-alpha line}. \textbf{Garavito-Camargo J.N}; Forero-Romero J.E;
    Dijkstra M. ApJ, 795, 120, 2014.
\end{enumerate}

{\Large Talks}\\


\indent \indent \textbf{12 Total, 7 International.}\\
\indent\indent The Royal Observatory of Edinburgh, UK, Coffee Talk, August 2019 \dag.\\
\indent\indent Durham University, UK, \href{http://astro.dur.ac.uk/cosmodwarfs/}{Small Galaxies Cosmic Questions}, August, 2019*.\\
\indent\indent Keck telescope, Journal Club, June 2019.\\
\indent\indent Magellanic Cloud Fest III, University of Arizona, May, 2019.\\
\indent\indent MPIA, Heidelberg, \href{http://www.mpia.de/homes/stellarhalos2018-loc/sh2018/index.html}{Stellar halos across the cosmos}, July 2018*.\\
\indent\indent JILA Seminar, JILA Institute, University of Colorado, December 2017.\\
\indent\indent STScI Galaxies Journal Club. December 2016.\\
\indent\indent LARIM, Cartagena, Colombia, October 2016*.\\
\indent\indent Magellanic Cloud Fest II, University of Arizona, March 2016.\\
\indent\indent EWASS, Geneve, Switzerland, July 2014*.\\
\indent\indent UNAM, Mexico City, Mexico, Nov 2013. \\
\indent\indent Centro de Investiaciones De Astronom\'ia CIDA, Merida, Venezuela, July 2013.\\

\indent\indent * \textit{Contributed talk in conferences.}
\indent\indent\ \dag \textit{Invited.} \\

{\Large Posters}\\

\indent\indent Potsdam, Germany. July 2017.\\


{\Large {Professional Experience}}\\

%indent\indent {\large Computational Skills}\\
%\indent\indent\indent Programming Languages: Python, C/C++.\\
% put links of this software
%\indent\indent\indent Tools: Git, \LaTeX, OpenMP, Mayavi.\\



\indent\indent Referee for \textit{MNRAS} 2019-present.\\ 
\indent\indent Co-organizer, \href{https://www.as.arizona.edu/diversity_coffee/}{Diversity Journal Club}, Steward Observatory, 2018-present\\
\indent\indent Writer for \href{https://astrobitos.org/}{Astrobitos}, 2018-present.\\ 
\indent\indent Proposal reviewer for \href{https://clubesdeciencia.co/}{Colombian Science Clubs}, 2018.\\
\indent\indent Local Organizing Committee of the \href{http://forero.github.io/AndeanCosmologySchool/}{Andean Cosmology School}, Universidad de Los Andes, 2015.\\
\indent\indent Organizer of the student astronomy seminar at the Planetarium of Bogot\'a, 2014-2015.\\



{\Large Outreach}\\

\indent\indent Creator, \href{https://astrocharlas.github.io/}{Spanish Outreach Lecture Series}, Steward Observatory, 2018-present\\
\indent\indent Writer for \href{https://astrobitos.org/}{Astrobitos}, 2018-present.\\ 
\indent\indent Classroom astronomer, NOAO Project \href{https://www.noao.edu/education/astro/}{ASTRO}, 2018-2019.\\
\indent\indent Discussion leader, NOAO \href{http://www.teenastronomycafe.org/}{Teen Astronomy Cafe}, 2018.\\
\indent\indent Mentor, Tucson Initiative for Minority Engagement in Science and Technology Program \href{https://lavinia.as.arizona.edu/~timestep/}{TIMESTEP}, 2016-2018.\\
\indent\indent Planetarium SpaceArt mentor for Children, Bogot\'a, Colombia, 2013-2014.\\
%\indent\indent Unconscious Bias Workshop, 2017. \\


{\Large Teaching Experience}\\

\indent\indent TA, Astronomy Tutoring for
Majors \& Minors Program. University of Arizona, Spring 2019. \\
\indent\indent Computational Physics, Teaching assistant, PHYS305. Univervsity of Arizona. Spring 2018.\\
\indent\indent Computational Tools: Universidad de los Andes, $1^{st}$ Semester-2015.\\
\indent\indent Computational Methods Laboratory: Universidad de los Andes, $1^{st}$ Semester-2015.\\
\indent\indent Computational Tools: Universidad de los Andes, $2^{nd}$ Semester-2014.\\
\indent\indent Computational Methods Laboratory: Universidad de los Andes, $2^{nd}$ Semester-2014.\\
\indent\indent TA, Computational Methods: Universidad de los Andes, $2^{nd}$ Semester-2014.\\
\indent\indent Computational Tools: Universidad de los Andes, $1^{st}$ Semester-2014.\\
\indent\indent Computational Methods Laboratory: Universidad de los Andes, $1^{st}$ Semester-2014.\\
\indent\indent TA, Computational Methods: Universidad de los Andes, $1^{st}$ Semester-2014.\\
\indent\indent Physics I Laboratory: Universidad de los Andes, $2^{nd}$ Semester-2013.\\


{\Large Observing Experience}\\

\indent\indent VATT telescope. Mt Graham, Arizona - 4 nights, 2016.\\
\indent\indent CIDA, Merida, Venezuela - 2 nights, 2013.\\



%{\Large Outreach talks:}\\

%\indent\indent Splendido retirement community Lecture Series, 02/28/20 \\


%\Large Languages}\\

%\indent\indent Spanish (Native), English (Fluent).

%\textsc{\Large Congresses \& Meetings:}\\
%$\left[6\right]$ Satellite galaxies and dwarfs in the local group. Leibniz Institute for Astrophysics, Potsdam, Germany. \indent \indent \ \ \ \ \ \
%\ \ \ \ \ \ \ \ \ \ \ \ \ \ \ \ \ \ \ \ \ \ \ \ \ \ \ \ \ \ \ \ \ \ \ \ \ \ \ \ \ \ \ \ \ \ \ \ \ \ \ \ \ \ \ \ \ \ \ \ \ \ \ 08/2014  \\
%$\left[5\right]$ European Week of Astronomy and Space Science EWASS.  \indent \ \ \ \ \ \ \ \ \ \ \ \ \ \ \ \ \ 07/2014 \\
%$\left[4\right]$ XIV Latin American Regional IAU Meeting (LARIM), Florianopolis, Brasil.  11/2013
%\normalsize {(Poster Contribution: 'Effects of the gas bulk Rotation on the morphology of the Ly $\alpha$ line')}\\
%$\left[3\right]$\large{ Fifth Venezuelan Astronomy Reunion (RVA), Merida, Venezuela. \indent \ \ \ \  \ \ \ \ \ \ \ \ \ \ \ \ 12/2012}
%\normalsize{(Talk Contribution: 'Effect of the rotation of distant galaxies on the Lyman-alpha emission line'.)}\\
%$\left[2\right]$\large{ Colombian Congress in Astronomy (COCOA), Bucaramanga, Colombia. \indent \ \ 11/2012}
%\normalsize{(Poster Contribution: Study Of The Dynamical Friction Time-Scale Using N-body Simulations.)}\\
%$\left[1\right]$\large{ XVII National Physics Congress, Santa Marta, Colombia. \indent \ \ \ \ \ \ \ \ \ \ \ \ \ \ \ \ \  \ \ \ \ \ \ \ \ \ \ \ \ \ \ \ \ \ \ \ \   /2009}\\


%\textsc{\Large Summer Schools \& Workshops attended:}\\
%\\
%\begin{enumerate}
%\setlength\itemsep{0em} 
%\item  \textit{Andean School on Cosmology}, University of Los Andes, 06/2015.\\
%\item \textit{CLUES meeting}, IAP Potsdam, Germany. \indent \ \ \ \ \ \ \ \ \ \ \ \ \ \ \ \ \ \ \ \ \ \ \ \ \ \ \ \ \ \ \ \ \ \ \ \ \ \ \ \ \ \ \ \ \ \ \ 08/2014\\
%\item \textit{GAIA Visualization workshop}, Viena, Austria, 07/2014.\\
%$\left[10\right]$ Vatican Observatory Summer School (VOSS).\\
%"Galaxies Near and Far, Young and Old". \indent \ \ \ \  \ \ \  \ \ \  \ \ \  \ \
% \ \ \ \ \ \ \ \ \ \ \ \ \ \ \ \ \ \ \ \ \ \ \ \ \ \ \ \ 06/2014 \\
%\item \textit{Galactic Dynamics in the Times of GAIA and other Great
%Surveys}, Mexico City, UNAM CU, 11/2013.\\
%$\left[8\right]$ \textsc{NEBULATOM}, "A Capacity Development Workshop for Latin\\
%American Astronomers on Emission Line Objects in the Universe". \indent \ \ \ \ \ \ \ \ \ \ \ \ \ \ \ \ \ \ 03/2013 \\
%Choron\'i, Venezuela.  \\
%$\left[7\right]$ III Venezuelan Astronomy School (EVA). Merida, Venezuela. \indent \ \ \ \ \ \ \ \ \ \ \ \ \ \ 12/2012\\
%\item \textit{Hippac Summer School on Astrocomputing}, University of California San Diego, 07/2012 \\
%\left[5\right]$ Data reduction in Cosmology School.  \indent \ \ \ \ \ \ \ \ \ \ \ \ \ \ \ \ \ \ \ \ \ \ \ \ \ \ \ \ \ \ \ \ \ \ \ \ \ \ \ \ \ \ 12/2011
%National Astronomical Observatory of Colombia.\\
%$\left[4\right]$ XVI Ciclo de cursos especiais em Astronomia.  \indent \ \ \ \ \ \ \ \ \ \ \ \ \ \ \ \ \ \ \ \ \ \ \ \ \ \ \ 10/2011
%National Observatory of Rio de Janeiro, Brasil. \\
%$\left[3\right]$ $1^{st}$ Workshop on Statistical Physics. \indent \ \ \ \ \ \ \ \ \ \ \ \ \ \ \ \ \ \ \ \ \ \ \ \ \ \ \ \ \ \ \ \ \ \ \ \ \ \ \ \ \ \ \ \ \ 08/2011
%National University of Colombia, Andes University, Bogot\'{a}, Colombia. \\
%\item \textit{School of Numerical Relativity}, National Astronomical Observatory of Colombia, 2011.\\
%\end{enumerate}
%$\left[1\right]$ Extragalactic Astrophysics School. \indent \ \ \ \ \ \ \ \ \ \ \ \ \ \ \ \ \ \ \ \ \ \ \ \ \ \ \ \ \ \ \ \ \ \ \ \ \ \ \ \ \ \ \ \ \ \ \ \ \ \ \ \ \ \ \ \ \ \ \ /2010
%National Astronomical Observatory of Colombia.\\

\end{document}
