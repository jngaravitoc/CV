\documentclass[letterpaper]{article}
\usepackage{enumerate}
\usepackage[hmargin=2cm,vmargin=2.5cm]{geometry}
\usepackage{hyperref}
\usepackage{url}
\begin{document}

\begin{center}
\textbf{\LARGE Juan Nicol\'as Garavito Camargo}\\
\textbf{\large Curriculum Vitae}\\
\end{center}


\textbf{\Large Contact Information:}\\

Steward Observatory, 933 North Cherry Av. Tucson, AZ.\\
\indent jngaravitoc[at]email [dot] arizona [dot] edu\\
\indent Personal web page:
\url{jngaravitoc.github.io/Garavito-Camargo}\\
\indent Github: \url{www.github.com/jngaravitoc} \\

\textbf{\Large Research Interests:}\\

Galaxy Dynamics, Local Group dynamics, Galaxy Formation and evolution.\\

\textbf{\Large Education:}\\

\textbf{University of Arizona}, Tucson, Arizona.\\
\indent Ph.D., Astronomy and Astrophysics (in progress).\\
\indent Advisor: Dr. Gurtina Besla\\

\textbf{University of Los Andes}, Bogot\'a, Colombia.\\
\indent M.Sc., Physics, 2015.\\
\indent Advisor: Dr. Jaime E. Forero-Romero\\

\textbf{National University of Colombia}, Bogot\'a, Colombia.\\
\indent B.Sc., Physics, 2013.\\
\indent Advisor: Dr. Rigoberto A. Casas Miranda\\

\textbf{\Large Fellowships, Grants and Awards:}\\

\indent University of Arizona theory travel grant, 2018. \\
\indent University of Arizona theory travel grant, 2016. \\
\indent IAU travel grant, 2016.\\
\indent McCarthy-Stoeger scholarship, 2015-2017.\\
\indent SpaceArt scholarship Bogot\'a Planetarium: For developing art and science material for children, 2014. \\
\indent Best project at the GAIA summer school held at Mexico City, November, 2013.\\
%2010 - $1^{st}$ Place Maloka (interactive Science Museum) contest to design a didactical physics experiment.\\


\textbf{\Large Publications:}\\

First author: 1, co-author: 4.\\

\begin{enumerate}
\item \textit{The influence of Sagittarius and the Large Magellanic Cloud on the stellar disc of the Milky Way Galaxy.} Laporte, Chervin F. P., Johnston, Kathryn V., Gómez, Facundo A., \textbf{Garavito-Camargo, Nicolas}., Besla, Gurtina, MNRAS, 481, 286L, 2018.

\item \textit{The Extremely Luminous Quasar Survey in the Sloan Digital Sky Survey Footprint. II. The North Galactic Cap Sample.} Schindler, Jan-Torge., Fan, Xiaohui., McGreer Ian D., Yang Jinyi., Wang, Feige., Green, Richard., \textbf{Garavito-Camargo, Nicolas} et al., ApJ, 863, 144S, 2018.

\item \textit{Modelling the gas kinematics of an atypical Ly$\alpha$ emitting compact dwarf galaxy.} Forero-Romero, Jaime E., Gronke, Max., Remolina-Gutiérrez, Maria Camila, \textbf{Garavito-Camargo, Nicolas}, Dijkstra M. MNRAS, 474, 12F, 2018.

\item \textit{Response of the Milky Way's disc to the Large Magellanic Cloud in a first infall scenario.} Laporte, C., Gomez, F., Besla,G., Kathryn V. Johnston \& \textbf{Garavito-Camargo, Nicolas}. MNRAS, 473, 1218L, 2018.

\item \textit{The impact of gas bulk rotation on the morphology of the Lyman-alpha line}.\textbf{Garavito-Camargo J.N} ,Forero-Romero J.E, Dijkstra M. ApJ, 795, 120, (2014)

\end{enumerate}

\textbf{\Large Talks and Posters:}\\

\indent MPIA, Heidelberg, July 2018*.\\
\indent Postdam, July 2017 $\dagger$.\\
\indent JILA Seminar, JILA Institute, University of Colorado, December 2017.\\
\indent STScI Galaxies Journal Club. December 2016.\\
\indent LARIM, Cartagena, Colombia, October 2016.*\\
\indent Magellanic Cloud Fest, University of Arizona, March 2016.\\
\indent EWASS, Geneve, Switzerland, July 2014.*\\
\indent UNAM, Mexico City, Mexico, Nov 2013. \\
\indent Centro de Investiaciones De Astronom\'ia CIDA, Merida, Venezuela , July 2013.\\

* \textit{Contributed talk in conferences.}
$\dagger$ \textit{Poster contribution.} \\

\textbf{\Large Observing Experience:}\\

\indent 4 Nights at the VATT telescope. Mt Graham Arizona.\\
\indent 2 Nights at CIDA, Merida, Venezuela.\\

\textbf{\Large Teaching Experience:}\\

\indent Computational Physics, Teaching assistant, PHYS305. Univervsity of Arizona. Spring 2018.\\
\indent Computational Tools: Universidad de los Andes, $1^{st}$ Semester-2015.\\
\indent Computational Methods Laboratory: Universidad de los Andes, $1^{st}$ Semester-2015.\\
\indent Computational Tools: Universidad de los Andes, $2^{nd}$ Semester-2014.\\
\indent Computational Methods Laboratory: Universidad de los Andes, $2^{nd}$ Semester-2014.\\
\indent TA, Computational Methods: Universidad de los Andes, $2^{nd}$ Semester-2014.\\
\indent Computational Tools: Universidad de los Andes, $1^{st}$ Semester-2014.\\
\indent Computational Methods Laboratory: Universidad de los Andes, $1^{st}$ Semester-2014.\\
\indent TA, Computational Methods: Universidad de los Andes, $1^{st}$ Semester-2014.\\
\indent Physics I Laboratory: Universidad de los Andes, $2^{nd}$ Semester-2013.\\

\textbf{\Large Community Activities:}\\

\indent Classroom astronomer, NOAO Project ASTRO, 2018-2019.\\
\indent Co-organizer, Diversity Journal Club, University of Arizona, Spring 2018.\\
\indent Writer for Astrobitos, 2018-.\\ 
\indent Discussion leader, NOAO Teen Astronomy Cafe, 2018.\\
\indent Mentor, Tucson Initiative for Minority Engagement in Science and Technology Program, TIMESTEP, 2016-2018\\
\indent Local Organizing Committee of the Andean Cosmology School, University of Los Andes, June 2015.\\
\indent Organizer of the student astronomy seminar at the Planetarium of Bogot\'a, 2014-2015.\\

\textbf{\Large Computational Skills}:\\

\indent Programming Languages: Python, C, C++.\\
\indent Tools: git, latex.\\
%visualization Software: TOPCAT, GLUE, Sphviewer.\\

\textbf{\Large Languages:}\\

\indent Spanish (native), English(Fluent).

%\textsc{\Large Congresses \& Meetings:}\\
%$\left[6\right]$ Satellite galaxies and dwarfs in the local group. Leibniz Institute for Astrophysics, Potsdam, Germany. \indent \indent \ \ \ \ \ \
%\ \ \ \ \ \ \ \ \ \ \ \ \ \ \ \ \ \ \ \ \ \ \ \ \ \ \ \ \ \ \ \ \ \ \ \ \ \ \ \ \ \ \ \ \ \ \ \ \ \ \ \ \ \ \ \ \ \ \ \ \ \ \ 08/2014  \\
%$\left[5\right]$ European Week of Astronomy and Space Science EWASS.  \indent \ \ \ \ \ \ \ \ \ \ \ \ \ \ \ \ \ 07/2014 \\
%$\left[4\right]$ XIV Latin American Regional IAU Meeting (LARIM), Florianopolis, Brasil.  11/2013
%\normalsize {(Poster Contribution: 'Effects of the gas bulk Rotation on the morphology of the Ly $\alpha$ line')}\\
%$\left[3\right]$\large{ Fifth Venezuelan Astronomy Reunion (RVA), Merida, Venezuela. \indent \ \ \ \  \ \ \ \ \ \ \ \ \ \ \ \ 12/2012}
%\normalsize{(Talk Contribution: 'Effect of the rotation of distant galaxies on the Lyman-alpha emission line'.)}\\
%$\left[2\right]$\large{ Colombian Congress in Astronomy (COCOA), Bucaramanga, Colombia. \indent \ \ 11/2012}
%\normalsize{(Poster Contribution: Study Of The Dynamical Friction Time-Scale Using N-body Simulations.)}\\
%$\left[1\right]$\large{ XVII National Physics Congress, Santa Marta, Colombia. \indent \ \ \ \ \ \ \ \ \ \ \ \ \ \ \ \ \  \ \ \ \ \ \ \ \ \ \ \ \ \ \ \ \ \ \ \ \   /2009}\\


%\textsc{\Large Summer Schools \& Workshops attended:}\\
%\\
%\begin{enumerate}
%\setlength\itemsep{0em} 
%\item  \textit{Andean School on Cosmology}, University of Los Andes, 06/2015.\\
%\item \textit{CLUES meeting}, IAP Potsdam, Germany. \indent \ \ \ \ \ \ \ \ \ \ \ \ \ \ \ \ \ \ \ \ \ \ \ \ \ \ \ \ \ \ \ \ \ \ \ \ \ \ \ \ \ \ \ \ \ \ \ 08/2014\\
%\item \textit{GAIA Visualization workshop}, Viena, Austria, 07/2014.\\
%$\left[10\right]$ Vatican Observatory Summer School (VOSS).\\
%"Galaxies Near and Far, Young and Old". \indent \ \ \ \  \ \ \  \ \ \  \ \ \  \ \
% \ \ \ \ \ \ \ \ \ \ \ \ \ \ \ \ \ \ \ \ \ \ \ \ \ \ \ \ 06/2014 \\
%\item \textit{Galactic Dynamics in the Times of GAIA and other Great
%Surveys}, Mexico City, UNAM CU, 11/2013.\\
%$\left[8\right]$ \textsc{NEBULATOM}, "A Capacity Development Workshop for Latin\\
%American Astronomers on Emission Line Objects in the Universe". \indent \ \ \ \ \ \ \ \ \ \ \ \ \ \ \ \ \ \ 03/2013 \\
%Choron\'i, Venezuela.  \\
%$\left[7\right]$ III Venezuelan Astronomy School (EVA). Merida, Venezuela. \indent \ \ \ \ \ \ \ \ \ \ \ \ \ \ 12/2012\\
%\item \textit{Hippac Summer School on Astrocomputing}, University of California San Diego, 07/2012 \\
%\left[5\right]$ Data reduction in Cosmology School.  \indent \ \ \ \ \ \ \ \ \ \ \ \ \ \ \ \ \ \ \ \ \ \ \ \ \ \ \ \ \ \ \ \ \ \ \ \ \ \ \ \ \ \ 12/2011
%National Astronomical Observatory of Colombia.\\
%$\left[4\right]$ XVI Ciclo de cursos especiais em Astronomia.  \indent \ \ \ \ \ \ \ \ \ \ \ \ \ \ \ \ \ \ \ \ \ \ \ \ \ \ \ 10/2011
%National Observatory of Rio de Janeiro, Brasil. \\
%$\left[3\right]$ $1^{st}$ Workshop on Statistical Physics. \indent \ \ \ \ \ \ \ \ \ \ \ \ \ \ \ \ \ \ \ \ \ \ \ \ \ \ \ \ \ \ \ \ \ \ \ \ \ \ \ \ \ \ \ \ \ 08/2011
%National University of Colombia, Andes University, Bogot\'{a}, Colombia. \\
%\item \textit{School of Numerical Relativity}, National Astronomical Observatory of Colombia, 2011.\\
%\end{enumerate}
%$\left[1\right]$ Extragalactic Astrophysics School. \indent \ \ \ \ \ \ \ \ \ \ \ \ \ \ \ \ \ \ \ \ \ \ \ \ \ \ \ \ \ \ \ \ \ \ \ \ \ \ \ \ \ \ \ \ \ \ \ \ \ \ \ \ \ \ \ \ \ \ \ /2010
%National Astronomical Observatory of Colombia.\\

\end{document}
